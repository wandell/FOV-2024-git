\subsection*{Cornea and Lens}

\be

\item The refractive index of air $\approx 1.000$, glass $\approx
1.520$, water $\approx 1.333$, cornea $\approx 1.376$.
% Wandell

\item Lens wavelength absorbance peaks 365-368nm.
( Said and Weale, 1959; Cooper and Robson, 1969)
% David Wooding

% \item Short-wavelength limit of visual sensitivity (human lens): 380nm

\item The macular pigment extends up to 8 deg from fovea.
% David Wooding

\item Axial length of the eye is about 25 cm (1 inch).
%Robert Boynton, Brian Brown

\item Axial chromatic aberration of the human eye over the visible
spectrum: 2 dipoters.
%(Howarth and Bradley, Vision Research 26, 361-366, 1986;
% Wald and Griffin 1947, JOSA, 37, 321-336).
% Peter Howarth

\item Corneal power is 43D ; lens power is 20 D (relaxed); whole eye 60 D;
change in from accommodation, 8 D.
% Source:  Davson, H. (Ed.) The Eye, vol. 4, page 105 (1962), Academic
% Press. This is the Gullstrand schematic eye.
% Brian Brown, L. Spillmann, Wandell

\item Corneal shape:  radius, 7.8 mm; diameter, 12mm.
% Brian Brown

\item Change in refractive power: about  3D for each mm change in axial length
% Brian Brown

\item Focal length of lens and cornea (relaxed accommodation): 0.017 m
% Brian Brown

\item [Macaque] focal length:  0.012 m $\approx$ 80 Diopters
% Wandell

\item [Macaque] 1 deg spans 0.000209 m
\comment{x/0.012m = tan(1 deg), x = 0.012* tan(1deg) =}
% Wandell

\ee

