\subsection*{Intensity Units}

\be

\item {\em Radiance} measures physical energy ($watts / {m ^2}$);
luminance units adjust radiance for visual wavelength sensitivity
(candelas per meter squared, denoted as $cd / {m^2}$).  Scotopic
luminance units correct for rod wavelength absorptions; Photopic
luminance units correct for a weighted combination of the $\Red$ and
$\Green$ cone wavelength absorptions.
% Wandell

\item The luminance level in $cd/m^2$ under several common sources:
starlight $\approx 10^{-3}$; moonlight $ \approx 10^{-1}$; indoor
lighting $\approx 10^2$; sunlight $\approx 10^5$.
%Source: Hood \& Finkelstein, Ch. 5 in some book from which I have an
%unlabeled reprint.  Maybe it's: Handbook of perception and human performance /
%editors, Kenneth R. Boff, Lloyd Kaufman, James P. Thomas. New York :
%Wiley, c1986.]
% Ben Backus

\item One Troland (Td) is produced on the retina when the eye is looking at a
surface of 1 $cd/m^2$ through a pupil of area 1 $mm^2$.  (Troland was a
color scientist and entrepreneur.)
%Steve Burns

%\item candelas per $m^2 \times ``area of pupil in'' mm^2$ = photopic
%trolands (Tds)

%
%    I am not sure I believe this.  Check a little more
%
% \item The luminous efficiency ratios of the R,G,B phosphor
% emissions of a typical CRT monitor: R:G:B = 3:6:1
% Jeff Mulligan

\item At a wide range of ambient intensities, the pupil diameter is
near 3.6 mm and thus the pupil area is $10 mm^2$.  Often, then, Tds =
$10 \times cd / {m^2}$)
% Chris Tyler

\item For a standard eye 1 Td produces 0.0035 $lumens /{m^2}$ on the retina.
%Steve Burns

\item In 'free' viewing to a scene or surface:  
retinal illuminance [in troland] = 
luminance of the scene [in candela/sq m] * pupil area [in sq mm] 
% Aart Kooijman

\item 1 Td is about 1 absorbed photon per cone per 100 msec
%  Stan Klein

\item  illuminance (lumens/$m^2$) = luminance $(cd/ m^2) \times \pi
\times (\sin A/2) ^ 2$, where A is the visual angle of the stimulus' diameter.
%Thomy Nilson

%\item retinal illuminance = incident luminance X pupil area in mm sqd X .0016
%This correction include's Judd's constant of .004.  (See Applied
%Optics 1983, p.3462)
% Thomy Nilsson

\item At 555nm, 1 photopic lumen = 1 scotopic lumen.
The unit 'lumen' can be replaced by any other luminance unit, such as
lux, candela, nit, Troland, apostilb, footlambert.
%(though you have to avoid this non-metric unit)
%Aart Koojiman

\item In a Maxwellian view condition: place a lux meter at 10 cm
behind the focus of the maxwellian focus (thus about in the middle of
the brains of the subject) retinal illuminance [in troland] = $10^-4
\times $ the illumination on the lux meter (full illuminated sensor)
0.1 log unit is 27\%
%Aart Koojiman

\ee
