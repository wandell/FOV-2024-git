\begin{figure}
\centerline{
  \psfig{figure=../09motion/fig/gregory.ps,clip= ,width=5.5in}
}
\caption[Size Illusion]{
{\em A visual illusion presented to patient SB.}  Most normal
observers see the men as having different size.  Patient SB saw the
men as having the same size.  }
\label{f9:sizeSB}
\end{figure}
Experimental measurements also suggest that motion and depth
perception are fundamentally different in these patients.  Gregory's
patient, SB did not experience the very powerful motion aftereffect
(see Chapter~\ref{chapter:Motion}.  He did not perceive the size
constancy illusion shown in Figure~\ref{f9:sizeSB}.  Most observers
see the human figure further in the distance as larger than the other
two.  The patient saw the three men in the Figure as having the same
size.


\comment{page 169 in Rock's Perception, from Gibson, 1950, The
Perception of the Visual World}
\begin{figure}
\centerline{
\psfig{figure=../10ill/fig/cylinders.ps,clip= ,width=5.5in}
}
\caption[Cylinders in perspective]{
{\em The cylinders are the same size.}  They appear to have different
size because we interpret them in their three-dimensional context.
When J. J. Gibson (1950) introduced this illusion, he said there are
no illusions.  I suppose he meant that when viewed as
three-dimensional objects, the appearance is correct.  }
\label{f10:cylinders}
\end{figure}
There are many illusions that trade on our depthful interpretation to
demonstrate how poor our judgment is on the printed page.  For
example, the cylinders in Figure \ref{f10:cylinders} are the same
size.

\comment{Get and read the Gregory and Heard paper in Perception
in which they experiment with the effect.}
\begin{figure}
\centerline{
 %\psfig{figure=../10ill/fig/cafeWall.ps,clip= ,width=5.5in}
}
\caption[Cafe Wall Illusion]{
{\em The Munsterberg or Cafe Wall illusion.}
}
\label{f10:cafeWall}
\end{figure}
The Munsteberg Illusion also discovered by R. L. Gregory and called
the {\em Cafe Wall Illusion} by him is shown in
Figure~\ref{f10:cafeWall}.  The horizontal lines are parallel,
although they appear to converge and diverge.  There is a surprising
subtlety to this illusion.  The distortion of the tiles depends on the
precise shade of gray in the cracks.  When the shade of gray is too
light or too dark, the illusion vanishes.  Tyler () and Morgan have
taken a crack (sorry) at modeling the phenomenon using simple linear
filters (Citations here).

\begin{figure}
\centerline{
%  \psfig{figure=../10ill/fig/kerstenCylinder.ps,clip= ,width=5.5in}
}
\caption[Kersten Cylinder]{}
\label{f10:kerstenCylinder}
\end{figure}

\begin{figure}
\centerline{
\psfig{figure=../10ill/fig/occludedNecker.ps,clip= ,width=5.5in}
}
\caption[Occluded Necker]{
Occlusion helps us to see the object.
The bits don't come together on their own.
}
\label{f10:occludedNecker}
\end{figure}

\begin{figure}
\centerline{
%  \psfig{figure=../10ill/fig/ramaShading.ps,clip= ,width=5.5in}
}
\caption[Rama Shading]{}
\label{f10:ramaShading}
\end{figure}
The power of this illusion, like so many others, interacts with our
perception of the three-dimensional shape of the object.  When the
image outlines change, we interpret the shading differently.  In one
case we interpret the image as a surface shaded by illumination; in
the other as a change in the surface reflectance function.

\comment{Classsics from Barbara Gillam article,
p. 165 in Sci Am by Rock.  Also, around page 160 in Rock's own book.
Also, see the colored image on page 152.  Perhaps I can make one like
it using the Mac and the entry to Stanford down Palm Drive.  Try to
connect it with Gregory's corner window description of the illusion.
Make the point that the signficance of these geometric illusions is
what they tell us about seeing objects, not their impact on the page.}

\begin{figure}
\centerline{
 %\psfig{figure=../10ill/fig/length.ps,clip= ,width=5.5in}
}
\caption[Geometrical Illusions]{}
\label{f10:length}
\end{figure}
