
%	SHORTEN!!!

\paragraph{A consistent explanation of velocity estimation errors.}
There are computational algorithms that can estimate
the physical motions of the experimental stimuli
from the image information.
Hence, observers errors in perceiving the physical
speed and direction of moving plaids
is due to the incorrect processing strategies
used by the visual system.
The visual system has failed to segregate contrast,
pattern and color from motion estimation.
Algorithms, such as the intersection-of-constraints
idea, do not include these factors in their representation;
models of visual speed estimation must.

What can we say about the visual system's processing strategies?
Velocity estimation is based on
neural signals that have already been interpreted
as moving forms.
Motion estimation's dependence
on contrast and pattern, then, occurs because the
segregation into moving forms depends on sophisticated use
of contrast and pattern.

%	Just stuck it in
In addition to an effect on perceived speed,
the stimulus contrast can have an effect on perceived directon.
Ferrara and Wilson (1990) and Stone, Watson and Mulligan (1990)
have shown that the perceived direction
of the motion plaid depends on the
relative contrast of the components.
The direction of motion of the plaid
appears biased toward the direction
of the more visible component.
%	Just stuck it in

\begin{figure}
\centerline{
 \psfig{figure=../09motion/fig/plaidMotion.ps,clip= ,height=3.5in}
}
\caption[Plaid Motion Depends on Components]{
Stone and Ferrera and Wilson's description of plaid motion
errors.
}
\label{f9:plaidMotion}
\end{figure}
To see how this might happen, consider the
perceived direction of a plaid pattern
with components at different contrasts.
The intersection-of-constraints idea tells us that
we can estimate the speed and direction of a plaid from
the speed and direction of each of its components,
as illustrated in the left panel of Figure \ref{f9:plaidMotion}.

We have seen, however, that
low contrast gratings appear to move slowly.
Suppose that the neural mechanisms responsible
for motion estimation of the plaid direction
depend on signals that have
already been contaminated by this contrast-dependent velocity
estimation.
Then, the perceived direction of the plaid
will also be in error.
The perceived direction will be biased towards
the direction of the higher contrast component.
This type of model is being
explored by Stone (1991) and
Ferrera and Wilson (199X).

