\subsection{Motion after effects}
The waterfall motion after effect described by Addams
was originally explained in terms of eye movements.
Helmholtz referred to the effect as a
{\em giddiness} and wrote that
\begin{quote}
... if the passenger gazing out of the coach should happen to fix his
attention constantly on a speck on the window, the aforesaid giddiness
will not be developed, although, as before he is aware of objects
flying past him without, however, making the movements necessary to
focus them.  Incidentally, moreover, when the eye is steadily focused
on a point that is stationary with respect to it, the images of moving
objects will be completely obliterated when the speed is such as is
needed for this illusion.  The only way to recognize them is by
pursuing them with the eyes for short distances. (Helmholtz,
Physiological Optics III, p. 248).
\end{quote}

In this case, Helmholtz made a mistake.
Plateau (1850) showed that motion after-effects can
be obtained using spiral stimuli involving no eye 
translational eye movements.
Exner (1875) demonstrated interocular transfer of the effect.
If one stares at a waterfall with one eye, and then
observes the mountain with the opposite eye, the mountain
still appears to move.
Anstis and Gregory (1964)
reported that when subjects fixate on a spot, and
a set of stripes move to the right, there is a motion after-effect,
contradicting Helmholtz' observation.

\begin{figure}
\centerline{
%  \psfig{figure=../09motion/fig/mae.ps,clip= ,width=3.5in,height=3.5in}
}
\caption[Motion After Effect Demonstration]{
When presented to each eye, these clockwise stimuli give
rise to a counter-clockwise percept.
The motion after-effect is opposite the central
perception, rather than the monocular signals.
From Anstis and Moulden.
}
\label{f9:mae}
\end{figure}
Anstis and Moulden (1970)
demonstrated a particularly interesting
form of the motion after effect.
Observers adapted to a stimulus whose binocular
appearance is quite different from the monocular
signals within each eye
(see also Zeevi and Geri, 1985).
Anstis and Moulden presented clockwise movements to each
eye, but when the stimuli are fused
the appearance is of counterclockwise motion.
The motion after-effect followed the appearance of the
binocular motion, not the monocular motion.

