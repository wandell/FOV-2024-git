There are many types of information we would like
to be able to extract from the motion flow field.
The example Gibson emphasized,
estimating observer motion relative to a still scene,
has been analyzed particularly
thoroughly in recent years
(Heeger and Jepson,  Tomassi and Kanade, Poelman and Kanade).

The problem is illustrated
in Figure \ref{f9:mfLanding} 
which illustrates a motion flow field of a pilot landing on a runway.
From the motion flow field, we would like to compute two sets
of parameters.
First, we would like to compute the observer's new position,
including any shift and change in orientation.
Second, we would like to compute the distance from
the observer to each point in the image.
These distances are called the image's {\em depth map}.

We will analyze how to estimate these properties from
a known motion flow field.
The main step we will consider
is how the motion flow field
is calculated from an observer's motion and the image depth map.
The inverse calculation, from the motion flow field to
observer motion and image depth, has been analyzed
in a number of different papers (Rieger and Lawton,
Heeger and Jepson, Tomassi and Kanade, Poelman and Kanade,
Tsai and Huang).
We will follow the Heeger and Jepson analysis here.

