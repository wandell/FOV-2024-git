\begin{figure}
\centerline {
\psfig{figure=../04retina/fig/XY.null.ps,clip=,height=3.5in}
}
\caption[Classification into X and Y types]{
The responses of two different cells, classified as X and Y types,
are shown.
The responses are the average rate of action potential
firings during one stimulus period.
The stimulus, a sinusoidal grating, was introduced at four
different spatial phase positions.
The X-cell on the left has two null positions and the Y-cell has none.
[Figure 12 from the Friedenwald.]
}
\label{f4:XY.null}
\end{figure}
Kuffler's center-surround discovery
is attractive because it defines
a common property, shared by nearly all retinal ganglion cells.
In most empirical investigations we are happy
to find general organizational
principle that describes our measurements.
There is some tension, however, between the need
to describe all of our measurements using a common
principle and the need to distinguish among
characteristics that are important in
classifying different types of neurons.
We have already seen from the morphology that one can
identify several different types of retinal ganglion cells.
Figure \ref{f4:XY.null}
shows measurements of retinal ganglion cell action potentials
that Enroth-Cugell and Robson (1966) used
to develop an electrophysiological classification of
cat retinal ganglion cells.

Enroth-Cugell and Robson's study was
the first to
use quantitative methods for classifying retinal ganglion cells.
Enroth-Cugell and Robson classified neurons based on their
response to one-dimensional sinusoidal gratings.
The data in the Figure \ref{f4:XY.null} show
responses to four different input stimuli.
The responses on the left plot the average rate
of action potentials during one stimulus period.
The rates are derived from many repeated
trials, yielding the smooth curves.
These responses are characteristic of
the class of cells that Enroth-Cugell and Robson
labeled {\em X-cells}
and the responses on the right are characteristic of 
cells they labeled as {\em Y-cells}.

The intensity profile of the one-dimensional
sinusoidal gratings can be described using
the equation
\begin{equation}
\label{e3:s0}
\intensityxy = \mean ( 1.0 + \contx \sin ( 2 \pi \fx x + \phix ) ) .
\end{equation}
The intensity of the light
at location $(x,y)$ is $\intensity (x,y) $,
the mean level of the light stimulus is $\mean$;
the peak contrast of the spatial modulation is $\contx$;
the spatial frequency of the spatial modulation is $f_x$;
and the phase of the spatial pattern is $\phix$.
The stimulus is one-dimensional because the right hand side
of Equation \ref{e3:s0} does not depend on the variable $y$.
The four sinusoidal stimuli shown in
Figure \ref{f4:XY.null} have the same mean level,
contrast, and spatial frequency;
they differ in their spatial phase, $\phix$.

Enroth-Cugell and Robson were particularly struck by
the fact that
for the X-cell there
are two stimulus phase positions at which
introducing the stimulus causes no change
in the neurons activity.
Enroth-Cugell and Robson called these stimulus phase positions
{\em null positions} and suggested that we classify
neurons based on the presence or absence of null positions.
This terminology is useful because it is closely related to 
the more quantitative analysis we will undertake later.
The phrase null position reminds us that the stimulus
is in the {\em null space} of the neuron's input-output function:
that is, when we use a grating in one of these
two spatial phase positions as input,
the output is zero.
Colloquially, we might say that X-cells are blind to
sinusoidal stimuli at these two phase positions.

Cells labeled as Y-cells don't have
null responses for any phase
position of the sinusoidal grating.
On the right of Figure \ref{f4:XY.null},
we see that the Y-cell generates
a transient response
for all of the phase positions.
The Y-cells respond to the grating in each phase position.
Also, notice that the responses
to gratings shifted by 180 degrees ($\pi$ radians)
is essentially the same.
Since a phase shift of 180 degrees simply changes the
sign of the contrast, $\contx$,
we can qualitatively
summarize these measurements by noting that
a Y-cell fails to distinguish between a contrast
pattern $\contx \sin ( 2 \pi \fx x ) $ and $ - \contx \sin ( 2 \pi \fx x ) $.

\begin{figure}
\centerline {
\psfig{figure=../04retina/fig/drifting.grating.ps,clip=,height=3.5in}
}
\caption[X and Y responses to drifting grating]{
The responses of an X- and a Y-cell to a drifting sinusoidal grating.
From E-C and R, 1966, figure 4.
}
\label{f4:XY.drifting.grating}
\end{figure}
There are several other features of
retinal ganglion cell responses that are correlated with
Enroth-Cugell and Robson's observations.
The existence of these correlated properties
adds support to the suggestion
that retinal ganglion cells can be classified
based on their responsiveness to other stimuli.
%	CITATION NEEDED
A number of authors noticed that when the sinusoidal
grating drifts across the receptive field of X-cells,
the response is modulated both above and below the
spontaneous firing rate so that the average response
remains unchanged.
For Y-cells the movement of a grating through the
receptive field causes a significant change in
the mean level of response.
This difference is illustrated in Figure \ref{f4:XY.drifting.grating}.

A second way in which the responses of X and Y-cells differ
was noted in a remarkable study performed by
Cleland, Dubin and Levick.
These authors recorded simultaneously from pairs of
neurons, one in the lateral geniculate and one in the retina.
They first located a neuron in the lateral geniculate
and determined the location of its receptive field.
They then used a second microelectrode to record
from a retinal ganglion cell whose receptive field
coincided with the lateral geniculate neuron's receptive field
{\em and} whose action potentials could be shown to induce responses
in the lateral geniculate neuron.
They found some neuron pairs
where every retinal ganglion cell action potential
was followed by a spike in the lateral geniculate neuron.
The correlation strongly suggested that the retinal
ganglion cell axon was a major input to the lateral
geniculate cell.

From these heroic experiments, Cleland, Dubin and Levick
made two striking observations.
The responses of neurons
in the lateral geniculate nucleus could be classified
into X and Y type responses much in the same
way as the responses of retinal ganglion cells.
Their measurements revealed that retinal ganglion
cells that the electrophysiological connections
were segregated by type:  retinal X-cells project
to X-cells in the lateral geniculate, and retinal Y-cells
project to Y-cells in the lateral geniculate.

Figure \ref{f4:conduction.time}
\begin{figure}
\centerline {
\psfig{figure=../04retina/fig/conduction.time.ps,clip=,height=3.5in}
}
\caption[Conduction Velocities of X and Y]{
The conduction velocities of X- and Y-type retinal ganglion
cells fall into two distinct groups.
Figure 8 from Cleland, Dubin and Levick which is reprinted as
figure XIX-44 in Rodieck's book
}
\label{f4:conduction.time}
\end{figure}
Second, by measuring the time delay between the retinal
action potential and the lateral geniculate action
potential, Cleland, Dubin and Levick could
measure the velocity of the spike along the axon.
They noticed that the conduction time along axons
connecting Y type neurons was much faster than
the conduction time along axons connecting X type neurons.
shows the distribution of conduction times.
All of the connections between Y-type neurons
were less than 3.2 msec, while all of the conduction
times between X-type neurons exceeded this value.

\subsection{Correlation of Electrophysiology and Morphology}

At the start of this chapter I noted that we can
be most confident in the functional significance of our
classification schemes when the criteria we use in morphology,
electrophysiology, and biochemistry converge.
The classification of photoreceptors
into rods and cones serves as an excellent
model of a well-developed
functional classification of neurons.

We now have a second impressive example of
the functional classification of neurons.
Retinal ganglion
cells classified as parasol on morphological
grounds have fast conducting axons and large receptive fields.
In the cat, they are often classified as Y-cells.
(Though in the primate even parasol cells are often linear.)
Cells classified as midget cells on
morphological grounds have conduct action potentials more
slowly and have smaller receptive fields.
And finally, this distinction among retinal ganglion cell
types is preserved through the lateral geniculate, where their
projections are found to be segregated.
The parasol cells project to the magnocellular layers
and the midget cells project to the parvocellular layers.
The classification of retinal ganglion cells
derived from the electrophysiological measurements,
morphological studies, and anatomical connections provide
strong evidence for the existence of separate
functional pathways.

Finally, if one needs further evidence, we can
turn to the direct experiments of Cleland, Levick and Wassle (1975).
These authors made extensive electrophysiological measurements
in selected patches of the cat's retina.
Remember that there are many fewer Y-cells than X-cells. 
Cleland et al.
attempted to record from every single
Y-cell in the small retinal patch.
Each time they found a neuron that was classified
as a Y-cell, they made a small electrolytic lesion,
marking the position of all the Y-cells they could find.
They then stained the region to examine the morphological
properties of the retinal ganglion cells.
They found a one-to-one correspondence between the
morphologically identified $\alpha$ cells
and the positions of their lesions describing
the locations of the electrophysiologically
determined Y-cells.

