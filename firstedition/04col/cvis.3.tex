%
%	References start here
%
\nocite{Baylor1973}

\newpage
\section*{Exercises}

\begin{enumerate}

\item  Our analysis of color encoding
begins with the color-matching experiment.
Make sure you can explain the highlights of this experiment.

 \begin{enumerate}

 \item  Describe a procedure to measure
the color-matching functions for
an observer who is using three primary lights.

 \item What constraints apply to your selection of primary lights?

 \item What restrictions must you obey when
you select the primary lights in a color-matching experiment?

 \item Suppose you and your colleague measure a color-match to
a test light, but you use different sets of primary lights.
What will be the relationship between
the color-matching intensities you find
and those that your colleague finds?

 \end{enumerate}

\item When the eye is adapted to a steady light,
the nervous system readjusts its visual sensitivity in a
variety of ways.
For example, when you walk into a dark theatre from
the outdoors, at first you cannot see well.
But, after some time, your visual system adjusts
and it becomes easy to see the dim lights.
Similarly, a light presented on a bright background is
difficult to see but the same light presented on
a weak background may be easy to see.

In the following questions, think about the
difference between a pair of lights that match
one another versus what the lights look like -- color-matching
versus color appearance.

 \begin{enumerate}

 \item Suppose that we establish a pair of foveal lights
as metamers by adjusting them to match on a zero (black)
background.
(Since the lights are viewed purely in the fovea, they
are matched by the cones.)
Now, suppose we view the metamers
on an intense red background.
Based on the theory that color matches are photopigment
matches, will the two lights continue to be metamers?

\item Neitz, Neitz and Jacobs et al. (1991) 
claimed that some color normal males change their matches
when the test lights are superimposed on red fields.
In a related article,
they argue that there are more than three types
of cones in the human eye (Neitz et al., 1993).
Read their articles and evaluate their claims.
\comment{
TITLE:           More than three different cone pigments among people with 
                   normal color vision.
AUTHOR:          Neitz, J (Department Cellular Biology, Medical College 
                   Wisconsin, 8701 Watertown Plank Road, Milwaukee, Wi 53226, 
                   Usa.); Neitz, M; Jacobs, G H
PUBLICATION:     Vision Research. 1993 33(1):117-122.
                 Vision Res.
}

 \item Suppose that, in fact,
lights continue to match when they are superimposed
upon various backgrounds.
Seen on the bright background, the two lights are only barely visible.
Will the lights still match when they are presented 
in the dark, against no background?
What if we  present the lights in the periphery where there are many rods?

 \item  When the two lights are seen on the bright background
and on the dim background, will their
{\em appearance} be unchanged?

 \end{enumerate}

\item Suppose we represent two lights by the
three-dimensional vectors that represent
each light's cone photopigment sensitivities, $\bf a$ and $\bf b$.
The vector difference between the
two representation of the two lights is $\bf d = \bf a  - \bf b $.
Finally, consider two lights $\bf m$ and $\bf n$ that also differ
by this same vector $\bf d = \bf m - \bf n$.

 \begin{enumerate}

 \item Suppose that we double the intensity of $\bf a$ and $\bf b$.
What happens to the vector representing each of the lights?
What happens to the vector representing the difference
between the scaled lights?

 \item  Suppose that we express the coordinates of these lights
in another color space obtained by applying
a linear transformation, $T$.
What will be the vector difference between $\bf a$ and $\bf b$
in the new color space?
What will be the vector difference between $\bf m$ and $\bf n$
in the new color space?

 \item  Do you think the two lights represented by
$\bf a$ and $\bf b$ will be
as discriminable as the lights $\bf m$ and $\bf n$?
Why or why  not?
Do you know of any experimental data to support your
claim?
Should we collect some?
%(MacAdam?)

 \end{enumerate}

\item For many practical applications, people wish to
use only two dimensions to describe colored lights.
Specifically, they wish to compare the direction of the
three-dimensional vectors and ignore the length of the vectors.
The reduction in dimension
of the representation is usually
done by introducing chromaticity coordinates via the following
formula.
Suppose the entries of the three-dimensional
color vector are $L,M,S$.
Then we define two chromaticity coordinates, $l$ and $m$ as

\[
l = \frac{L}{L + M + S}
\]
\[
m = \frac{M}{L + M + S}.
\]

 \be

 \item Show that two vectors with color representations
that differ by a scale factor have the same chromaticity coordinates.

 \item 
Consider the following four lights
$(1,1,0),(0.7,1.0,0.3),(0.3,1.0,0.7),(0,1,1)$.
These lights are weighted mixtures of
the two components, $(1,1,0)$ and $(0,1,1)$.
Compute the chromaticity coordinates of 
lights that are formed as weighted sums of the lights.
Plot them on a graph whose axes are $l$ and $m$.

 \item Compute the general formula for the chromaticity of
a pair of lights formed as the mixture
\[
a (L,M,S) + b (L',M',S')
\]

 \item Challenge.  The chromaticity coordinates of this mixture
of the two lights describe a set of points on the chromaticity
diagram that depend on the weights, $a$ and $b$.
Call these points, $(l(a,b), m(a,b))$,
and prove that they always fall on a straight line.

 \ee

\begin{table}
 \caption{Photopigment Absorption Sensitivities (Source: Smith and Pokorny)}
 \label{t3:ncones}
 \begin{center}
  \begin{tabular}{||r|r|r||} \hline 
$\Red$ Cones & $\Green$ Cones & $\Blue$ Cones \\ \hline
0.004249 & 0.004602 & 0.174419 \\ \hline
0.008655 & 0.009716 & 0.364341 \\ \hline
0.015893 & 0.018921 & 0.662791 \\ \hline
0.023446 & 0.031705 & 0.906977 \\ \hline
0.030212 & 0.047814 & 1.000000 \\ \hline
0.034461 & 0.063667 & 0.918605 \\ \hline
0.041385 & 0.086167 & 0.802326 \\ \hline
0.062785 & 0.130657 & 0.693798 \\ \hline
0.102282 & 0.189210 & 0.468992 \\ \hline
0.162392 & 0.267706 & 0.279070 \\ \hline
0.263572 & 0.397597 & 0.166667 \\ \hline
0.424233 & 0.596778 & 0.096899 \\ \hline
0.618411 & 0.810534 & 0.046512 \\ \hline
0.775138 & 0.944515 & 0.023256 \\ \hline
0.885759 & 1.000000 & 0.011628 \\ \hline
0.956412 & 0.989772 & 0.003876 \\ \hline
0.995909 & 0.925850 & 0.003876 \\ \hline
1.000000 & 0.809000 & 0.000000 \\ \hline
0.967113 & 0.653030 & 0.000000 \\ \hline
0.896459 & 0.478650 & 0.000000 \\ \hline
0.796696 & 0.318844 & 0.000000 \\ \hline
0.672069 & 0.194068 & 0.000000 \\ \hline
0.531393 & 0.110458 & 0.000000 \\ \hline
0.380960 & 0.058553 & 0.000000 \\ \hline
0.257120 & 0.029660 & 0.000000 \\ \hline
0.159559 & 0.014319 & 0.000000 \\ \hline
0.091581 & 0.007159 & 0.000000 \\ \hline
0.048308 & 0.003324 & 0.000000 \\ \hline
0.025806 & 0.001534 & 0.000000 \\ \hline
0.012431 & 0.000767 & 0.000000 \\ \hline
0.006294 & 0.000256 & 0.000000 \\ \hline
  \end{tabular}
 \end{center}
\end{table}


\item Table~\ref{t3:ncones} lists estimates of
the proportion of photons absorbed
per second for unit intensity lights
at different wavelengths in the human eye.

 \begin{enumerate}

 \item How many photons will be absorbed during
one second to a light at 500nm and 5 units of intensity?
What about a light at 600 nm and 10 units of intensity?
Answer for both receptor classes.

 \item How many photons will be absorbed in each
receptor class when we present the superposition of the two lights?
Again, answer for both receptor classes.

 \item How would you set the intensities of the 500nm and 600nm
lights so that the absorptions to these lights
equal the absorptions to a unit intensity 550nm light?

\item Can you set the intensities of the 500nm and 600nm
lights so that the absorption rate matches a
400nm light at 10 units of intensity? 

 \end{enumerate}

\item Now, suppose you are studying the color-matching performance
of a dichromat, a person with
only the $\Red$ and $\Green$ cones.
We can summarize the properties of the two
receptor system using some simple drawings.

 \begin{enumerate}

 \item Make a graph whose x-axis is the rate of
absorptions by the first photoreceptor and whose
y-axis is the rate of absorption by the second photoreceptor.
On the graph plot the rate of absorptions to
each of the unit-intensity monochromatic lights.

 \item On the same graph,
plot the number of absorptions during one second
to a 500nm
light at 0.5 units of intensity and 2 units of intensity. 
Plot the number of absorptions during one second
to a 600nm light at 0.5 units of intensity.

 \item  On the same graph,
plot the number of absorptions in one second
to mixtures of
500nm and 600nm when their respective intensities are
(0.2,0.8), (0.5,0.5), and (0.8,0.2).

 \end{enumerate}

\item Answer the following questions about scotopic sensitivity.

 \be

 \item  Suppose you study the wavelength sensitivity
of an observer under scotopic viewing conditions.
At the end of the experiment you discover that the observer
was wearing tinted contact lenses.
The observer has to go on an extended holiday to Mars, but
is willing to leave his contact lenses behind.
What measurements do you need to make to correct
your estimate of the observer's wavelength sensitivity?

 \item  There are some intensity ranges in which both rods and
cones actively respond to lights.
At those intensity levels, human observers are still
trichromatic, even though there are four active receptor classes.
How can this be?

 \item Suppose we adjust a pair of lights so that they
are metamers under scotopic vision.
Will they be metamers under photopic vision?

 \item Suppose we adjust a pair of lights so that they
are metamers under photopic vision.
Will they be metamers under scotopic vision?

 \item A yellow daisy and a blue lilac
may be perceived to be equally bright
under scotopic conditions.  Purkinje noticed that these
lights are not equally bright under photopic conditions,
where the yellow flower is perceived to be much brighter.
This phenomenon is called the {\em Purkinje shift}.
Explain the phenomenon.

 \ee

\item Answer the following questions on the limits of color-matching.

 \be

 \item  Use a computer drawing
program to make a pattern of fine yellow and blue lines.
Make sure that the colors in the lines look blue and yellow
when you are close to the monitor.
Step away from the monitor three or four meters.
What happens to the color appearance of the lines?
Try the same with white and black lines.
What happens to their appearance?

 \item What optical effects could be playing a role in the
experiment in part (a)?

 \item Given what you know about the optics of the eye,
do you think we will obtain the same color-matching functions
if we repeat our experiments using a 10 cycles per degree sinusoidal
pattern rather than a uniform 2 deg spot?
What qualitative expectations do you have about the
10 cycles per degree sinusoidal pattern experiments?

 \item Suppose you establish a metameric match.
Then you put on a pair of sunglasses.
Will the metameric match be preserved?
Describe why or why not.

 \item As we age, the wavelength transmissivity of our cornea
and lens changes.
What effect will this have on the color-matching functions?

 \ee

\item In an abstract for a meeting, Knoblauch and McMahon (1993)
described a test of a cure for dichromacy.
The idea, which is also found in Tom Cornsweet's book, is simple.
Dichromats should wear a tinted lens over one eye.
This changes the spectral absorption of the photopigments in that
eye, providing enough information in the photopigment absorptions
to permit discrimination of lights that were previously identical
at the photopigments.

Now consider a dichromatic subject, Mr. X, as described by
James Clerk Maxwell
\begin{quote}
By furnishing Mr X. with a red and a green glass, which he could
distinguish only by their shape, I enabled him to make judgements in
previously doubtful cases of a colour with perfect certainty.  I have
since had a pair of spectacles constructed with one eye-glass red and
the other green.  These Mr X. intends to use for a length of time, and
he hopes to acquire the habit of discriminating red from green tints
by their different effects on his two eyes.  Though he can never
acquire our sensation of red, he may then discern for himself what
things are red, and the mental process may become so familiar to him
as to act unconsciously like a new sense. (J.C. Maxwell, Trans. Roy.
Soc. Edin. (1855), v. 21, p. 275-298, reprinted in Scientific Papers,
W.D. Niven (ed.), Dover, New York).
\end{quote}

Do you agree with Maxwell that Mr. X's experience of color
would be the same if he were to ear the tinted-lens glasses?
Knoblauch and McMahon, who are protanopes,
thought that the ability to perform discriminations did change
when wearing the glasses.
Even if you are not a dichromat, try this idea for yourself.
Do you agree with their conclusions?

\end{enumerate}  %All Questions

