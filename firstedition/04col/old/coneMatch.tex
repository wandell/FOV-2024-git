


If we plot the number of 500nm photons necessary
to match some number of 659 nm photons, we should find
a straight line relationship, as in Figure \ref{f3:scot.homogeneity}.
The existence of such a straight
line relationship would confirm homogeneity component of linearity
at a stage prior to the generation photocurrent,
presumably at the level of the photopigment absorption.
The slope of the straight line tells us the amount of
500 nm light necessary to have the same effect as each unit of
659 nm light.
For these wavelengths, we expect
a straight line with a slope of $1 / 9.3 = 0.108$, as illustrated
in the made-up data in part (b) of Figure \ref{f3:univariance}.
