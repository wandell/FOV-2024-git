\section{The Perceptual Organization of Color}
In this section, I will review some of the methods for describing the
perceptual organization of color appearance.  Specifically, we will
review the relationship between different colors and some of the
systems for describing color appearance.  In addition to the
implications this organization has for understanding the neural
representation of color appearance, there are also many practical
needs for descriptive systems of color appearance.  Artists and
designers need ways to identify and specify the color appearance of a
design.  Further, they need ways of organizing colors and finding
interesting color harmonies.  Engineers need to assess the appearance
and discriminability of colors used to highway signs and to label
parts, packaging, and software icons.

Language provides us with a useful start at organizing color
appearance.  Spoken English in the U.S. consists of eleven color terms
that are widely and consistently used\footnote{White, black, red,
green, yellow, blue, brown, purple, pink, orange, gray.}  While the
number of terms used differs across cultures, there is a remarkable
hierarchical organization to the order in which color names appear.
Cultures with a small number of basic color names always include
white, black and red.  Color terms such as purple and pink enter later
(Berlin and Kay, 1969; Boynton and Olsen, 1987).

Color names are a coarse description of color experience.  Moreover,
names list, but do not organize, color experience.  Thus, they are not
helpful when we consider issues such as color similarity or color
harmony.  A more complete organization of color experience is based
on the three perceptual attributes called: {\em hue, saturation}, and
{\em brightness}.  Hue is the attribute that permits a color to be
classified as red, yellow, green, and so forth.  Saturation describes
a color's similarity to a neutral gray or white.  A gray object with a
small reddish tint has little saturation, while a red object, with
little white or gray, is very saturated.  An object's brightness tells
us about the relative ordering of the object on the dark to light
scale.

Based on psychological studies of the similarity of colored patches
with many different hues, saturations and brightnesses, the artist
Albert Munsell created a book of colored samples.  The appearance of
the samples is organized with respect to hue, saturation and
brightness.  Furthermore, the colored samples are spaced in equal
perceptual steps.  Munsell organized the samples within his book in
terms using a cylindrical organization as shown in
Figure~\ref{f8:munsell}.  The {\em Munsell Book of Colors} is
published and used as a reference in many design and engineering
applications.
\begin{figure}
\centerline{
  \psfig{figure=../09col/fig/munsell.ps,clip= ,width=5.5in} }
\caption[Munsell Organization]{ {\em The Munsell Book of Colors} is a
collection of colored samples organized in terms of three perceptual
attributes of color.  The samples are arranged using a cylindrical
geometry with respect to these attributes.  The main axis of the
cylinder codes lightness; the distance from the center of the
cylinder to the edge codes the Munsell property called value
(saturation); the position around the circumference of the cylinder
codes the Munsell property called chroma (hue).  } 
\label{f8:munsell}
\end{figure}

Perceptually, both saturation and brightness can be arranged using a
linear ordering from small to large; hue, however, does not follow a
linear ordering.  So, Munsell organized lightness along the main axis
of the cylinder, and saturation as the distance from the center of the
cylinder to the edge.  The circular hue dimension was mapped around
the circumference of the cylinder. The Munsell Book of Colors notation
is widely used in industry and science.

Munsell developed a special notation to refer to each of the samples
in his book.  To distinguish his notation from the colloquial usage,
Munsell substituted the word {\em value} for lightness and the word
{\em chroma} for saturation.  He retained the word hue, apparently
finding no adequate substitute.  In the Munsell notation, the words
hue, chroma and value have specific and technical meanings.
Each colored paper is described using a three-part syntax of hue
chroma/value.  For example, 3YR 5/3 refers to a colored paper with the
hue called 3YR, the chroma level 5, and the value level 3.

The Munsell Book was created before the CIE color standards described in
Chapter~\ref{chapter:wavelength}.  With the advent of the CIE measurement
standard, based on the color-matching functions, there was
a need for a method to convert the Munsell representation into the CIE
standard representation.  A committee of the Optical Society, led by
Nickerson, Newhall and Evans, measured the CIE values of the Munsell
samples in the published book, and the Munsell Corporation agreed to
produce the colored samples to measurement standards defined by the
Optical Society of America.  The new standard for the Munsell Book,
based on CIE values rather than pigment formulae, is called the {\em
Munsell Renotation System}.  Calibration tables that describe the
color measurements of the Munsell Book samples are tabulated, for
example, Wyszecki and Stiles (1982).

\subsection*{Opponent-Colors} One of the most remarkable and
important insights about color appearance is the concept of {\em
opponent-colors}, first described by E. Hering (1878).  Hering
pointed out that there is a powerful psychological relationship
between the different hues.  While some pairs of hues can coexist in
a single color sensation, others cannot.  For example, orange is
composed of red and yellow while cyan is composed of blue and green.
But, we never experience a hue that is simultaneously red and green.
Nor do we experience a color sensation that is simultaneously blue
and yellow.  These two hue pairs, red-green and blue-yellow, are
called {\em opponent-colors}.

There is no physical reason why these two opponent-colors pairs
must exist.  That we
never perceive red and green, while we easily perceive red and yellow,
must be due to the neural representation of colors.  Hering argued
that opponent-colors exist because the sensations of red and green are
encoded in the visual pathways by a single pathway.  The excitation of
the pathway causes us to perceive one of the opponent-colors;
inhibition of the pathway causes us to perceive the other.

Hering made his point forcefully, and extended his theory to explain
various other aspects of color appearance, as well.  But, his insights
were not followed by a set of quantitative studies.  Perhaps for this
reason, his ideas languished while the colorimetrists used
color-matching to set standards for all of modern technology.  This is
not to say Hering's work was forgotten.  Colorimetrists who thought
about color appearance invariably turned to Hering's insights.  In a
well-known review article, the eminent scientist D. B. Judd, wrote

\begin{quote}
The Hering (1905) theory of opponent colors has come to be fairly well
accepted as the most likely description of color processes in the
optic nerve and cortex.  Thus this theory reappears in the final stage
in the stage theories of von Kries-Schrodinger (von Kries, 1905;
Schrodinger, 1925), Adams (1923, 1942) and Muller (1924, 1930).  By
far the most completely worked out of these stage theories is that of
Muller. ... There is slight chance that all of the conjectures are
correct, but, even if some of the solutions proposed by Muller prove
to be unacceptable, he has nevertheless made a start toward the
solution of important problems that will eventually have to be faced
by other theorists.  [Handbook of Experimental Psychology, 1951,
p. 836].
\end{quote}

\subsection*{Hue Cancellation. }
Several experimental observations, beginning in the mid-1950s, catapulted
opponent-colors theory from a special-purpose model, known only to
to color specialists, to a central idea in vision science.

The first was a behavioral experiment that defined a procedure for
measuring opponent-colors, the {\em hue cancellation} experiment.  The
hue cancellation experiment was developed in a series of papers by
Jameson and Hurvich (1955; 1957).  By providing a method of
quantifying the opponent-colors insight, Hurvich and Jameson made the
idea accessible to other scientists, opening a major line of inquiry.

In the hue cancellation experiment, the observer is asked to judge
whether a test light appears to be, say, reddish or greenish.  If the
test light appears reddish, the subject adds green light in order to
cancel precisely the red appearance of the test.  If the light appears
greenish, then the subject adds red light to cancel the green
appearance.  The added light is called the {\em canceling} light.
Once the red or green hue of the test light is canceled, the test
plus canceling light appear yellow, blue, or gray.  The same
experiment can be performed to measure the blue-yellow opponent-colors
pairing.  In this case the subject is asked whether the test light
appears blue or yellow, and the canceling lights also appear blue and
yellow.

Figure~\ref{f8:hueCancel} shows a set of hue cancellation measurements
obtained by Jameson and Hurvich (1955, 1957).  Subjects canceled the
red-green or blue-yellow color appearance of a series of spectral
lights.  The vertical axis shows the relative intensity of the
canceling lights, scaled so that when equal amounts of these lights
are superimposed the result did not appear, say, red or green.  The
canceling lights always have positive intensity, but the intensity of
the green and blue canceling lights are plotted as negative to permit
you to distinguish which canceling light was used.
\begin{figure}
\centerline{
 \psfig{figure=../09col/fig/hueCancel.ps ,clip= ,width=5.5in} }
\caption[Hue Cancellation]{ {\em Measurements from the hue
cancellation experiment.} An observer is presented with a
monochromatic test light.  If the light appears red then some amount
of a green canceling light is added to cancel the redness.  If the
light appears green, then a red canceling light is added to cancel
the greenness.  The horizontal axis of the graph measures the
wavelength of the monochromatic test light, and the vertical axis
measures the relative intensity of the canceling light.  The entire
curve represents the red-green appearance of all monochromatic
lights.  A similar procedure is used to measure blue-yellow.
(Source: Hurvich and Jameson, 1957).  
}
% Hurvich and Jameson % Psych Rev. 1957, vol. 64, no. 6, pp. 384-404
% Figure 4.  
\label{f8:hueCancel} \end{figure}

To what extent can we generalize from red-green measurements using
monochromatic lights to other lights?  The answer to this question we
must evaluate the linearity of the hue cancellation experiment.  If the
experiment is linear, we can use the data in Figure~\ref{f8:hueCancel}
to predict whether any test light will appear red or green
(blue-yellow) since all lights are the sum of monochromatic lights.  If the
experiment is not linear, then the data represent only an interesting
collection of observations.

To evaluate the linearity of the hue cancellation experiment, one can
perform the following experiment: Suppose the test light $\testi{1}$
appears neither red nor green, and the test light $\testi{2}$ appears
neither red nor green.  Does the superposition of these two test
lights, $\testi{1} + \testi{2}$, also appear neither red nor green?
In general, the hue cancellation experiment fails this test of
linearity.  If we superimpose two lights, neither of which appears red
or green, the result can appear red.  If we add two lights neither of
which appears blue or yellow, the result can appear yellow.  Hence,
the hue cancellation studies are a useful benchmark.  But, we need a
more complete (nonlinear) model before we can apply the hue
cancellation data in Figure~\ref{f8:hueCancel} to predict the
opponent-colors appearance of polychromatic test lights (Larimer et
al. 1975; Burns et al., 1984; Ayama et al. 1989; Chichilnisky, 1995).

\subsection*{Opponent-colors measurements at threshold} 

In addition to color appearance judgments, one can also demonstrate
the presence of essential opponent-colors signals behaviorally by
{\em color test-mixture experiments}.  These color experiments are
direct analogues of the pattern-mixture experiments I reviewed in
Chapter~\ref{chapter:space} (see Figures~\ref{f6:testMixtureGraph}
and \ref{f6:testMixtureData}).  Measurements from a color
test-mixture experiment are shown in Figure~\ref{f8:opponentB}.  The
axes represent the absorption rates of the $\Red$ and $\Green$ cones.
Each point represents a combination of absorption rates at detection
threshold.  The smooth curve fit through the data points is an
ellipse.  \begin{figure} \centerline{
 \psfig{figure=../09col/fig/opponentB.ps ,clip= ,width=5.5in}
}
\caption[Color ellipse]{
{\em Color test-mixture experiments demonstrate opponent-colors
processes.}  The axes measure percent change in cone absorption
rates for the $\Red$ and $\Green$ cones.  The points show the cone
absorptions rates at detection threshold measured using different
colored test lights.  The smooth curve is an ellipse fit through the
data points.  The mixture experiment shows that the $\Red$ and
$\Green$ cone signals cancel one another, so that lights that excite a
mixture of $\Red$ and $\Green$ cones are harder to see than lights that
stimulate just one of these two cone class (Source: Wandell, 1986).  }
\label{f8:opponentB}
\end{figure}

The intersection of the ellipse with the horizontal axis, shows the
relative $\Red$ cone absorption rate at detection threshold.  The
intersection of the ellipse with the vertical axis shows the relative
$\Green$ cone absorption rate at detection threshold.  The shape of
the ellipse shows that signals that stimulate the $\Red$ and $\Green$
cones simultaneously are less visible than signals that stimulate only
one or the other.  At the most extreme points on the ellipse, the cone
absorptions of the $\Red$ and $\Green$ cones are more than five times
the rate required to detect a signal when each cone class is
stimulated alone.  The poor sensitivity to mixtures of signals from
these two cone types shows that the signals must oppose one another.
The cancellation of threshold level signals from the $\Red$ and
$\Green$ cones, as well as between the $\Blue$ cones and the other two
classes (not shown), have been observed in many different laboratories
and under many different experimental conditions. (e.g., Boynton, et
al., 1964; Mollon and Polden, 1977; Pugh, 1976, 1979; Stromeyer et
al., 1985; Sternheim, 1979; Wandell and Pugh, 1980ab).


In addition to demonstrating opponent-colors, these threshold data
reveal a second interesting and surprising feature of visual
encoding.  Two neural signals that are visible when they are seen
singly become invisible when they are superimposed.  It seems odd
that the visual system should be organized so that plainly visible
signals can be made invisible.  From the figure we can see that this
is a powerful effect, suppressing a signal that is more than five
times threshold.  This observation tells us that in many operating
conditions absolute sensitivity is not the dominant criterion.  The
visual pathways can sacrifice target visibility in order to achieve
the goals of the opponent-colors encoding.

\subsection*{Opponent Signals in the Visual Pathways.  }
In addition to these two types of behavioral evidence, there is also
considerable physiological evidence that demonstrates the existence of
opponent-colors signals in the visual pathway.

In a report that gained widespread attention, Svaetichin (1956)
measured the responses of three types of retinal neurons in a fish.
He reported that the electrical responses were qualitatively
consistent with Hering's notion of the opponent-colors
representation.  In two types of neurons, the electrical response
increased to certain wavelengths of light and decreased in response
to other wavelengths, paralleling the red-green and blue-yellow
opponency in color perception\footnote{ At first, it was thought that
these responses reflected the activity of the the cones.  Subsequent
investigations showed that the responses were from horizontal cells
(MacNichol and Svaetichin, 1958).}.  The electrical response of a
third set of neurons increased to all wavelengths of light, as in a
black-white representation.  Shortly after Svaetichin's report, De
Valois and his colleagues established the existence of
opponent-colors neurons in the lateral geniculate nucleus of nonhuman
primates.  There is now a substantial literature documenting the
presence of color opponent-signals in the visual pathways (e.g.
DeValois et al., 1958, DeValois 1965; DeValois, G. Jacobs and I.
Abramov,1966; Wiesel and Hubel, 1966; Gouras, 1968; Derrington, et
al., 1984; Lennie and Krauskopf Cortex paper, 1991).

The resemblance between the psychological organization of
opponent-colors measured in the hue cancellation experiment and the
neural opponent-signals suggests a link from the neural responses to
the perceptual organization.  To make a convincing argument for the
specific connection between opponent-colors and a specific set of
neural opponent-signals, we must identify a linking hypothesis.  The
hypothesis should tell us how we can predict appearance from the
activity of cells, and conversely how we can predict the activity of
these cells from appearance.

A natural starting place is to suppose that there is a population of
neurons whose members are excited when we perceive red and inhibited
when we perceive green.  From the linking hypothesis, we predict that
neurons in this population will be unresponsive to lights that appear
neither red nor green.  There are two spectral regions that appear
neither red nor green to human observers: one near 570nm and a second
near 470nm.  To forge a strong link between appearance and neural
response, we can ask whether the candidate neural population fails to
respond to lights that appear neither red nor green.  Then, we might
search for a second population that fails to respond to lights that
appear neither blue nor yellow.

This question was studied by DeValois and his collaborators in the
lateral geniculate nucleus of the monkey.  In their studies, DeValois
and his colleagues studied the response of neurons to monochromatic
stimuli presented on a zero background.  They found a weak
correspondence between the neutral points of individual neurons and
the perceptual neutral points (DeValois, et al.  1966).  More recently,
Derrington, Krauskopf and Lennie (1984) measured the responses of
lateral geniculate neurons using contrast stimulus presented on a
moderate, neutral background.  They estimated the input to these
neurons from the different cone classes and confirmed the basic
observations made by DeValois and his colleagues.

Derrington et al. reported that parvocellular neurons could be
classified into two groups of neurons.  One population of neurons
receives opposing input from the $\Red$ and $\Green$ cones.  The panel
on the left of Figure~\ref{f8:dkl} shows my estimate of the spectral
sensitivity of this group of parvocellular neurons.  For these neurons
wavelengths near 570nm are quite ineffective.  But, there is a great
deal of variation within this cell population making it difficult
to be confident in the connection.  Moreover, these neurons do not
show a second zero-crossing near 470nm that would parallel the human
opponent-colors judgments in the hue cancellation experiment.

A second population of lateral geniculate neurons receives input from
the $\Blue$ cones and an opposing signal from a combination of the
$\Red$ and $\Green$ cones.  For these neurons, wavelengths near 500nm
are quite ineffective.  The panel on the right of Figure~\ref{f8:dkl}
shows my estimate of the spectral sensitivity of this group of
parvocellular neurons.
\begin{figure}
\centerline{
 \psfig{figure=../09col/fig/dkl.ps ,clip= ,width=5.5in} 
}
\caption[Derrington,Krauskopf and Lennie]{ {\em Opponent-signals
measured in lateral geniculate nucleus neurons.} These spectral
response curves are inferred from the measured responses of lateral
geniculate neurons to many different colored stimuli presented on a
monitor.  The vast majority of lateral geniculate neurons in the
parvocellular layers can be divided into two groups based on their
response to modulations colored lights.  One group of neurons
receives an opponent contribution from the $\Red$ and $\Green$ cones
alone (panel a).  The second group of neurons receives a signal of
like sign from the $\Red$ and $\Green$ cones, and an opposing signal
from the $\Blue$ cones (panel b) (Source: Derrington et al., 1984).
} \label{f8:dkl} 
\end{figure}

There was less order in the opponent-color signals of the
magnocellular neurons.  Many magnocellular units seemed to be driven
by a difference between the $\Red$ and $\Green$ cones.  A few
parvocellular units and a few magnocellular units were driven by a
positive sum of the two signals from these two cone types.

The spectral responses of these neural populations suggest that there
is only a loose connection between the signals coded by these neurons and the
perceptual coding into opponent-hues; it is unlikely that the
excitation and inhibition causes our perception of red-green and
blue-yellow.  One difficulty is the imperfect correspondence between
the neural responses and the hue cancellation measurements.  The
second difficulty is that there is no substantial population of
neurons representing a white-black signal.  This is a very important
perceptual dimension which must be carried in the lateral geniculate
nucleus signals.  Yet, no clearly identified group of neurons can be
assigned this role\footnote{Some authors have suggested that a single
group of lateral geniculate neurons codes a white-black sensation for
high spatial frequency patterns and a red-green sensation for low
spatial frequency patterns.  While this is an interesting hypothesis,
notice that the authors have abandoned the idea that there is a
specific color sensation associated with the response of lateral
geniculate neurons.  Instead, they suppose that the perceived hue
depends on the pattern of neural activation (Ingling and Martinez,
1984; Derrington, et al., 1984).}.

\subsection*{Decorrelation of the Cone Absorptions} The
opponent-signals measured in the lateral geniculate nucleus probably
represent a code used by the visual pathways because of its
properties in communicating information from the retina to the brain.
The psychological opponent-colors coding may be a consequence of the
coding strategy used to communicate information from the retina to
the cortex.  What reason might there be for using an opponent-signals
coding?

One reason to use an opponent-signal representation has to do with the
efficiency
of the visual encoding.  Because of the overlap of the $\Red$ and
$\Green$ cone spectral sensitivities, the absorption rates of these
two cone types are highly correlated.  This correlation represents an
inefficiency in the visual coding of spectral information.  As I
described in Chapter~\ref{chapter:multiresolution}, decorrelating the
signals can improve the efficiency of the neural representation.

We can illustrate this principle by working an example, parallel to
the one in Chapter~\ref{chapter:multiresolution}.  Consider the cone
absorptions to a set of surfaces.  Because of the overlap in spectral
sensitivities, the cone absorptions between, say, the $\Red$ and
$\Green$ cones will be correlated.  To remove the correlation, we
create a new representation of the signals consisting of the $\Red$
cone absorptions alone, and a weighted combination of the the $\Red$,
$\Green$, and $\Blue$ cone absorptions.  We will choose the weighted
combination of signals so that the new signal is independent of the
$\Red$ cone absorptions.  As we reviewed in the earlier chapter, by
decorrelating the cone absorptions before they travel to the brain, we
make effective use of the dynamic range of the neurons transmitting
the information (Buchsbaum and Gottschalk, 1986).

The graphs in Figure~\ref{f8:decor}ab show examples of the correlation
of the cone absorptions for a particular set of surfaces and
illuminant.  These plots represent the cone absorptions from light
reflected by a Macbeth ColorChecker viewed under mean daylight
illumination.  The correlations shown in these two plots are typical
of natural images: the $\Red$ and $\Green$ cone absorptions are highly
correlated (panel a); the $\Green$ and $\Blue$ cone absorptions are
also correlated (panel b).
\begin{figure}
\centerline{
 \psfig{figure=../09col/fig/decor.ps ,clip= ,width=5.5in}
}
\caption[Cross Correlation]{
{\em Absorptions in the three cone classes are highly correlated.}
The correlation between cone absorptions can be measured using
correlograms.  In this figure, correlograms are shown of the cone
absorptions from the surfaces in the Macbeth ColorChecker illuminated
by average daylight.  (a) A correlogram of the $\Red$ and $\Green$
cone absorptions.  (b) A correlogram of the $\Red$ cone absorptions
plotted versus a weighted sum of the cone absorptions that is
decorrelated from the $\Red$ cone absorptions, $ -.59 \Red + 0.8
\Green - .12 \Blue$. }
\label{f8:decor}
\end{figure}

As described in Chapter~\ref{chapter:multiresolution}, we decorrelate
the signals derived from the cone absorptions by forming new signals
that are weighted combinations of the cone absorptions.  There are
many linear transforms of the cone absorptions that could serve to
decorrelate these absorptions.  One such transformation is represented
by the following three linear equations\footnote{ This decorrelation
is based on the singular value decomposition of the cone absorptions.
},
% This is the matrix u in the decor sensor analysis in decorrelate.m
\begin{eqnarray}
O_1(\lambda) & = &  1.0L(\lambda) +  0.0M(\lambda) + 0.0S(\lambda) 
 \nonumber \\
O_2(\lambda) & = & -0.59L(\lambda) + 0.80M(\lambda) + -0.12S(\lambda) 
 \nonumber \\
O_3(\lambda) & = &  -0.34L(\lambda) + -0.11M(\lambda) + 0.93S(\lambda)
\label{e8:decor}
\end{eqnarray}
or, written in matrix form, 
\begin{equation}
\left (
 \begin{array}{ccc}
   & O_1(\lambda) & \\
   & O_2(\lambda) & \\
   & O_3(\lambda) & \\
 \end{array} \right ) =
\left (
 \begin{array}{rrr}
     1.00  &  0.00 &  0.00 \\
    -0.59  &  0.80 & -0.12 \\
    -0.34  & -0.11 &  0.93 \\
 \end{array}
\right )
\left (
 \begin{array}{ccc}
   & \Red(\lambda) & \\
   & \Green(\lambda) & \\
   & \Blue(\lambda) & \\
 \end{array}
\right ) \nonumber
\label{e8:decor2}
\end{equation}

The new signals, $O_i(\lambda)$, are related to the cone absorptions
by a linear transformation.  These three signals are decorrelated with
respect to this particular collection of surfaces and illuminant.

The spectral sensitivity of the three decorrelated signals are shown
in Figure~\ref{f8:decorSensors}.  The two opponent spectral
sensitivities are reminiscent of the hue cancellation measurements and
the opponent-signals measured in the lateral geniculate nucleus.  One
of the sensors has two zero-crossings, near 570nm and 470nm.  A second
sensor has one zero-crossing near 490nm.  The third sensor has no
zero-crossings, as required for a white-black pathway.  The similarity
between the decorrelated signals, the opponent-signals in the lateral
geniculate nucleus, and the hue cancellation experiment suggest a
purpose for opponent-colors organization.  Opponent-colors may exist
to decorrelate the cone absorptions and provide an efficient neural
representation of color (Buchsbaum and Gottschalk, 1986; Derrico and
Buchsbaum, 1991).
\begin{figure}
\centerline{
 \psfig{figure=../09col/fig/decorSensors.ps,clip= ,width=5.5in}
}
\caption[Decorrelated Sensors]{
{\em The spectral responsivity of a set of color sensors whose
responses to the Macbeth ColorChecker under mean daylight are
decorrelated.}  The spectral sensitivities of these sensors resemble
the spectral sensitivities of lateral geniculate neurons and the color
appearance judgments measured in the hue cancellation experiment.  }
\label{f8:decorSensors}
\end{figure}

The opponent-colors representation is a universal property of human
color appearance, just as the need for efficient coding is a simple
and universal idea.  We should expect to find a precise connection
between opponent-colors appearance and neural organization in the
central visual pathways.  The hue cancellation experiment provides us
with a behavioral method of quantifying opponent-colors organization.
Hue cancellation measurements establish a standard for
neurophysiologists to use when evaluating opponent-signals in the
visual pathways as candidates for the opponent-colors representation.
Opponent-colors organization is a simple and important idea; pursuing
its neural basis will lead us to new ideas about the encoding of
visual information.

\subsection*{Spatial Pattern and Color}
Figures~\ref{f3:albers} and \ref{f8:contrast} show that the color
appearance at a location depends on the local image contrast, that is,
the relationship between the local cone absorptions and the mean image
absorptions.  The targets we used to demonstrate this dependence are
very simple spatial patterns, squares or lines, with no internal
spatial structure of their own.  In this section, we will review how
color appearance can also depend on the spatial structure, such as the
texture or spatial frequency, of the target itself.

Figure~\ref{f8:squarewave} shows two squarewaves composed of
alternating blue and yellow bars.  One squarewave is at a higher
spatial frequency than the other.  The average signal reflected from
the regions containing the squarewaves is the same, that is, these are
pure contrast modulations about a common mean field.  If you examine
the squarewaves from a close distance, you will see that bars in the
squarewave patterns are drawn with the same ink.  If you place this
book a few meters away from you, say across the room, the color of the
bars in the high spatial frequency pattern will appear different from
the color of the bars in the low spatial frequency pattern.  The bars
in the high spatial frequency pattern will appear to be light and dark
modulations about the green average.  The bars in the low spatial
frequency pattern will continue to look a distinct blue and
yellow\footnote{You can also alter the relative color appearance of
the patterns by moving the book rapidly up and down.  You will see
that the low frequency squarewave retains its appearance while the
high frequency squarewave becomes a green blur.}.
\begin{figure}
\centerline{
  \psfig{figure=../09col/fig/squarewave.ps,clip= ,width=5.5in}
}
\caption[Squarewave Color illusion]{
{\em Color appearance covaries with spatial pattern.}  The bars
printed in these two squarewaves are the same.  Yet, whether the bars
appear the same or not depends on their spatial frequency which you
can control by altering the viewing distance.  Also, you can influence
the color appearance in the two patterns by moving the book rapidly up
and down while you look at the patterns.  (Source: Wandell, 1993).  }
\label{f8:squarewave}
\end{figure}

Poirson and Wandell (1993) used an asymmetric color-matching task to
study how color appearance changes with spatial frequency of the
squarewave pattern.  Subjects viewed squarewave patterns whose bars
were colored modulations about a neutral gray background; that is, the
average of the two bars comprising the pattern was equal to the mean
background level.  Subjects adjusted the appearance of a 2 degree
square patch to have the same color appearance as each of the bars in
the pattern.

Two qualitative observations stood out in this study.  First, spatial
patterns of moderate and high spatial frequency patterns (above 8 cpd)
appear mainly light-dark, with little saturation.  Thus, no matter
what the relative cone absorptions of a high spatial frequency target,
the target appeared to be a light dark variation about the mean level.
Second, the spatially asymmetric color appearance matches are not
photopigment matches.  This can be deduced from the first observation:
Because of axial chromatic aberration, moderate frequency squarewave
contrast patterns (4 and 8 cpd) cannot stimulate the $\Blue$ cones
significantly.  Yet, subjects match the bars in these high frequency
patterns using a 2 deg patch with considerable $\Blue$ cone contrast.
The asymmetric color-matches are established at neural sites central
to the photoreceptors.

Poirson and I explained the asymmetric spatial color matches using a
pattern-color separable model.  In this model, we supposed that the
color appearance of the target was determined by the response of three
color mechanisms, and that the response of each mechanisms was
separable with respect to pattern and color.  We derived the spatial
and spectral responsivities of these pathways from the observers
color-matches; the estimated sensitivities are shown in
Figure~\ref{f8:oPathways}.

Interestingly, the three color pathways that we derived from the
asymmetric matching experiment correspond quite well to the
opponent-color mechanisms derived from the hue cancellation
experiment.  One pathway is sensitive mainly to light-dark variation;
this pathway has the best spatial resolution.  The other two pathways
are sensitive to red-green and blue-yellow variation.  The blue-yellow
pathway has the worst spatial resolution.  Granger and Heurtley
(1973), Mullen (1985) and Sekiguchi et al., (1993ab) made
measurements that presupposed the existence of opponent-color pathways
and estimated similar pattern sensitivities of the three mechanisms.
Notice that the derivation of the opponent-colors representation in this
experiment did not involve asking the observers any questions about
the hue or saturation of the targets.  The observers simply set color
appearance matches;  the opponent-colors mechanisms were needed to
predict the color matches.
\begin{figure}
\centerline{
  \psfig{figure=../09col/fig/colorCsf.ps ,clip= ,width=5.5in}
}
\caption[Opponent pathway sensitivity:  color appearance]{
{\em Estimates of the pattern-color separable sensitivity of pathways
mediating color appearance.}  By measuring spatially asymmetric
color-matches, it is possible to deduce the pattern and color
sensitivity of three visual mechanisms that mediate color appearance
judgments.  The pattern and wavelength sensitivity of a light-dark,
red-green, and blue-yellow mechanism derived from experimental
measurements are shown here.  (Source:  Poirson and Wandell, 1993).
}
\label{f8:oPathways}
\end{figure}

One of the more striking aspects of opponent-colors representations is that
the apparent spatial sharpness, or focus, of a color image depends
mainly on the sharpness of the light-dark component the image; apparent
sharpness depends very little on the spatial structure of the opponent-color
image components.  This is illustrated in the three images shown in
Figure~\ref{f8:imageCompress}.  These images were created by
converting the original image, represented as three spatial patterns
of cone absorptions, into three new images corresponding to a
light-dark representation and two opponent-colors representations.
The image in Figure~\ref{f8:imageCompress}a shows the result of
spatially blurring the light-dark component and then reconstructing
the image; the result appears defocussed.  The images in
Figure~\ref{f8:imageCompress}bc show the result of applying the same
spatial blurring to the red-green and blue-yellow opponent-colors
representations and then reconstructing.  These images look spatially
focused, though their color appearance has been changed somewhat.

\begin{figure}
\centerline{
 \psfig{figure=../09col/fig/compress.ps,clip= ,width=5.5in}
}
\caption[Color image compression]{
{\em The apparent spatial sharpness (focus)
of a color image depends mainly on the light-dark
component of the image, not the opponent-colors components.}  A
colored image was converted to a light-dark, red-green and blue-yellow
representation.  To create the three images, the light-dark (a),
red-green (b), or blue-yellow (c) components were spatially blurred
and then the image was reconstructed.  The light-dark image looks
defocused, but the same amount of blurring does not make the other two
images look defocused.  (Source: H. Hel-Or, personal communication).
}
\label{f8:imageCompress}
\end{figure}

We can take advantage of the poor spatial resolution of the
opponent-colors representations when we code color images for image
storage and transmission.  We can allocate much less information about
the opponent-colors components of color images without changing the
apparent spatial sharpness of the image.  This property of human
perception was important in shaping broadcast television standards and
digital image compression algorithms.  As a quantitative prediction,
we should expect to find that neurons in the central visual pathways
that represent light-dark information should be able to represent
spatial information at much higher resolution than neurons that code
opponent-colors information.  Consequently, we should expect that the
largest fraction of central neurons encode light-dark, rather than the
other two opponent-colors signals.

The differences between the light-dark encoding and the
opponent-colors encoding are of great perceptual significance.
Consequently, several authors have studied hypotheses based on the
idea that opponent-colors signals and light-dark signals are found in
separate areas of the brain.  In the final section of this chapter, we
will consider some of the evidence concerning the representation of
color information in the visual cortex.
