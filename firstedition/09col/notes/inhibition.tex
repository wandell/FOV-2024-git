The opponent-colors representation supposes that excitation and
inhibition are equally useful neural responses.  This should not seem
surprising to you since throughout this volume we have always supposed
that changes from the mean level of activity are a significant measure
of neural response.  But, there was a trend in early among early
physiologists to consider only increases in neural responses as
meaningful, and at the time Svaetichin's results arrived the notion
that inhibition was a meaningful signal was relatively new.  Hering's
opponent-colors insight, and the discovery of a neural counterpart to
this insight, has played an important role in establishing the
importance of inhibition as a meaningful physiological
signal~\footnote{Hurvich (1969) has argued that the need for accepting
inhibitory signals was a key reason that opponent-colors theory met
resistance among physiologists.  For example, as late as 1965,
R. L. DeValois wrote
\begin{quote}
We find no ground for the assumption that only increases in firing are
of importance.  This sort of thinking has led many investigators to
distort or misinterpret many findings about visual system functions.
[p. 152, Cold Spring Harbor Symp. 1965].
\end{quote}
}

\subsection*{Linking Opponent-Signals and Opponent-Colors}
