\section{Receptive Fields in Primary Visual Cortex}

The receptive fields of neurons in area V1 are qualitatively
different from those in the lateral geniculate nucleus.
For example, lateral geniculate neurons
have circularly symmetric receptive fields,
but most V1 receptive fields do not.
Unlike lateral geniculate neurons, some neurons in area V1
respond well to stimuli moving in one direction but fail to respond
to stimuli moving in the opposite direction.  
Some area V1 neurons are binocular,
responding to stimuli from both eyes.  
These new receptive field properties
must be related to the visual computations performed within the
cortex such as the analysis of form and texture,
the perception of motion, and the estimation of stereo depth.
We might expect that these new receptive field properties
have a functional role in these visual computations.

Much of what we know about cortical receptive fields comes from Hubel
and Wiesel's measurements during their 25 year collaboration.  
Others had accomplished the difficult feat of recording 
from cortical neurons first; 
but the initial experiments
used diffuse illumination, say turning
on the room lights, as a source of stimulation.  As we have seen,
pattern contrast is an important variable in the retinal neural
representation; consequently, cortical cells respond poorly to diffuse
illumination (von Baumgarten, and Jung, 1952).
Hubel and Wiesel made rapid progress
in elucidating the responses of cortical neurons by using stimuli of
great relevance to vision and by being extremely insightful.
Hubel and Wiesel's papers chart a remarkable series of
advances in our understanding of the visual cortex.  Their studies
have defined the major ways in which area V1 receptive fields differ
from lateral geniculate nucleus receptive fields.  Their qualitative
methods for studying the cortex continue to dominate experimental
physiology (Hubel and Wiesel, 1959, 1962, 1968, 1977; Hubel, 1982).

Hubel and Wiesel recorded the activity of cortical neurons
while displaying patterned stimuli, mainly line segments
and spots, on a screen that was imaged through
the animal's cornea and lens onto the retina.
As the microelectrode penetrated the visual cortex, they presented line
segments whose width and length could be adjusted.  
First, they varied the
position of the stimulus on the screen, searching for the neuron's
receptive field.  Once the receptive field position was established,
they measured the response of the neuron to a lines,
bars and spots presented individually.

One important goal of their work
was to classify the cortical neurons based on
their responses to the small collection of stimuli.
They sought classifications that represented the neurons'
receptive field properties and that also helped to clarify
the neurons' function in seeing.  
Classification of the receptive field types was an important
theme when we considered the responses of retinal ganglion cells as
well.  
It is of great current interest to try to understand whether
the classifications of cortical neurons and retinal neurons can be
brought together to form a clear picture of this entire section of the
visual pathways.

A second important aspect of characterizing cortical neurons
is to measure the
transformation from pattern contrast stimulus to firing activity. 
We used linear systems methods to design experiments and create
quantitative models of this transformation for retinal
ganglion cells.
Linearity is an important
idea when applied to cortical receptive fields, too.
The most important
application of linearity is Hubel and Wiesel's classification of
cortical neurons into two categories, called {\em simple} and {\em
complex}.  
This classification is based, in large part, on an informal
test of linearity (Skottun et al., 1991).  
As Hubel writes, ``For the most part, we can
predict the responses of simple cells to complicated shapes from their
responses to small-spot stimuli (Hubel, 1988, p. 72).'' 
Complex cells,
on the other hand, do not satisfy superposition.  
The response obtained by
sweeping a line across the cell's receptive field can
not be predicted accurately
from the responses to individual flashes of a line.

\subsection*{Orientation selectivity}
Since simple cells are approximately linear, we can measure their
receptive fields using the methods described in
Chapter~\ref{chapter:retina}.  Simple cell receptive fields consist of
adjacent excitatory and inhibitory areas, as illustrated in
Figure~\ref{f5:orientedRF}.
Simple cells have {\em oriented} receptive fields and
hence they respond to
stimuli in some orientations better than others.  
This receptive field
property is called {\em orientation selectivity}.  The orientation of
the stimulus the evokes the most powerful response is called the
cell's {\em preferred orientation}.

Orientation selectivity of cortical neurons is a new receptive field
property.  Lateral geniculate neurons and retinal neurons have
circularly symmetric receptive fields and they respond almost
equally well to
all stimulus orientations.  Orientation selective neurons are found
throughout layers 2 and 3, though they
are relatively rare in the
primary inputs within layer 4C.

\begin{figure}
\centerline {
\psfig{figure=../06cortex/fig/orientedRF.ps,clip=,width=5.5in}
}
\caption[Orientation Theory]{
{\em Orientation selective receptive fields} can be created by summing
the responses of neurons with non-oriented, circularly symmetric
receptive fields.  The receptive fields of three hypothetical neurons
are shown.  Each hypothetical
receptive field has an adjacent excitatory and inhibitory
region.  (a) and (c) illustrate that the degree of
orientation selectivity can vary depending on
the number of neurons combined along the main axis.}
\label{f5:orientedRF}
\end{figure}
Figure~\ref{f5:orientedRF} shows several
orientation selective linear receptive fields and how these
might be constructed from the outputs of lateral geniculate neurons.
The simple cell receptive fields consist of 
adjacent excitatory and inhibitory regions that are longer
in one direction than the other.
The main axis of the receptive fields defines the preferred
orientation;
stimuli oriented along the main axis of these receptive
fields are more effective at exciting or inhibiting the
cell than stimuli in other orientations.
The figure shows the
excitatory regions as resulting from the combined output of
neurons with excitatory centers and the inhibitory regions
resulting from the combined output of neurons with inhibitory centers\footnote{
In principle, one might construct an oriented receptive field
from the outputs of a single line of lateral geniculate neurons.
But, recall that the receptive fields of lateral geniculate
neurons have a weak opposing surround.
The inhibitory and excitatory regions of the
cortical neurons often are more nearly balanced in their effect.
Hence, I have constructed these regions by combining the outputs
from separate groups of neurons.}.

By comparing the three panels in
Figure~\ref{f5:orientedRF} you will see that
receptive fields sharing a common preferred orientation
can differ in a number of other ways.
Panels (a) and (b) show two receptive fields
with the same preferred orientation
but different spatial arrangements
of the excitatory and inhibitory regions.
Panels (a) and (c) show two receptive fields
with the same preferred orientation and arrangement of excitatory and
inhibitory regions, but
differing in the overall length of the receptive field.
The neuron with the longer receptive field will respond well
to a narrower range of stimulus orientations than the
neuron with the shorter receptive field.

Complex cells also show orientation selectivity.
Complex cells are nonlinear, so to explain the behavior of
complex cells, including orientation selectivity, will require 
more complex models than the simple sums of neural outputs
used in Figure~\ref{f5:orientedRF}.

The preferred orientation of neurons
varies in an orderly
way that depends on the neuron's position within the cortical sheet.
Figure~\ref{f5:orientation} shows
the preferred orientation of a collection of neurons
measured during a single, long, tangential penetration
through the cortex.
In any small region of layers 2 and 3,
the preferred orientation is similar.
As the electrode passes tangentially through the cortical
sheet, the preferred orientation changes systematically,
varying through all angles.
Figure \ref{f5:orientation}a shows an
extensive set of measurements of preferred orientation
made during a single tangential penetration (Hubel and Wiesel, 1977).
The change in preferred orientation is very systematic as the
electrode passes tangentially through the cortex.
Upon later review, Hubel and Livingstone (1984) noted that
during these measurements there were certain intervals during
which the receptive field orientation was ambiguous.
Figure~\ref{f5:orientation}b shows a second
penetration in which regions with no
preferred receptive field orientation are identified.
As we shall see, Hubel and Livingstone also report that the
regions lacking orientation selecitivity
coincide with locations in layers 2 and 3 cortex where
an enzyme called {\em cytochrome oxidase} is present in high density.
However, there is some debate whether these measurements 
represent true differences in the receptive fields
of individual neurons, or whether they represent 
differents in the distribution of activity
in local collections of neurons 
(O'Keefe et al., 19XX; Leventhal, et al., 19XX;).
\begin{figure}
\centerline {
 \psfig{figure=../06cortex/fig/orientation.ps,clip=,width=5.5in}
}
\caption[Orientation Columns in Visual Cortex]{
{\em The preferred orientation of neurons in area V1}
measured during a single tangential penetration.
The horizontal axis shows the distance along the tangential
penetration and the vertical axis shows the orientation
of the receptive field.
% Functional architecture of macaque visual cortezx. Proc. R. Soc.
% Ser. B. v. 198, p. 1-59, The Ferrier Lecture, Fig. 16.
%Taken in fact rom figure 37 in Bishop article (layers 2 and 3).
The data in panel (a) were reported by Hubel and Wiesel (1977).
The data in panel (b) are from a similar experiment
by Hubel and Livingstone (1984).
In this graph open rectangles denote
locations where responses to all orientations are equal.
%M. S. Livingstone and D. H. Hubel, Anatomy and physiology of a color
%system in the primate visual cortex.  The J. of Neuroscience. v. 4
%no. 1 p. 309-356.
%figure 5 (layers 2 and 3).)
}
\label{f5:orientation}
\end{figure}

The alternative interpretation is based on measurements
of the spatial organization
of cortical regions with common orientation preference.
Obermayer and Blasdel (1993)
measured regions with a common orientation preference
using a high resolution optical imaging method.
In this method, a voltage sensitive dye is applied to cortex.
Local neural activity causes
reflectance changes in the dye, and these can be
visualized by reflecting light from the exposed cortex.
By stimulating with visual signals in different orientations
and measuring the changes in reflectance,
Obermayer and Blasdel (1993) visualized regions
with common orientation preference;
by stimulating with images in originating in different eyes,
they could identify ocular dominance columns
(see also Hubel and Wiesel, 1977).

\begin{figure}
\centerline{
%  \psfig{figure=../06cortex/fig/isoOrientation.ps,clip= ,width=5.5in}
}
\caption[Iso-orientation lines and ocular dominance columns]{
{\em Regions with common orientation preference}
are shown as the gray lines in this contour plot.
The dark lines show the boundaries of ocular dominance columns.
At the edges of the ocular dominance columns, regions with
common orientation are arranged in parallel lines that
are nearly perpendicular to the ocular dominance columns.
These lines converge to singular points located near the
center of the ocular dominance columns.
(Source:  Obermayer and Blasdel, 1993).
% Their figure5b, page 4123.
}
\label{fig:isoOrientation}
\end{figure}
Figure~\ref{fig:isoOrientation} represents
Obermayer and Blasdel's (1993) measurements 
as a contour plot.
Regions with common orientation preference are shown
as gray iso-orientation lines, and
the boundaries of the ocular dominance columns are shown
as dark lines.
The Figure shows that
the variation in preferred orientation is synchronized with
the variation in ocular dominance.
A full range of preferred orientations
takes place within about 1 mm of the cortex,
about equal to one ocular dominance column.
Near the edges of the ocular dominance columns,
the iso-orientation lines
are arranged in linear, parallel strips
extending roughly 0.5 - 1 mm.
These linear strips are oriented nearly perpendicular to the edge
of the ocular dominance edge.
In the middle of the ocular dominance columns, the
iso-orientation lines converge toward single points
called {\em singularities}.
In these regions, neurons with receptive fields with
different preferred orientations are brought close to one another,
and they may also be the position
of the high density of cytochrome oxidase (Blasdel, 1992).  %J. Neuroscience.
These regions will have high metabolic
activity since, for any stimulus orientation
some of the neurons in the region will be active.
This is an alternative explanation of the
colocation of regions of high density cytochrome oxidase and 
regions of reduced orientation selectivity of the neural response.

There are a number of broad questions that remain unanswered
about the orientation selectivity in the visual cortex.
First, we might
ask how are the receptive field properties of cortical neurons
constructed from the cortical inputs?
Figure~\ref{f5:orientedRF} shows
that we can explain orientation selectivity theoretically
since combining signals from
center-surround neurons with adjacent receptive field locations
results in an oriented receptive field.
But, there is no empirical counterpart to this theoretcal
explanation.
Second, the regularity of the iso-orientation contours
shows that the orientation preferences of neurons
is created in an highly regular and organized pattern.
What are the rules
for making the interconnections that lead to this 
spatial organization of orientation selectivity?
What functional role to they have in perceptual processing?
Is this spatial organization essential for neural computations,
or is it merely a convenient wiring diagram for an area whose
output is communicated to other processing modules?

\subsection*{Direction selectivity}
Hubel and Wiesel (1968) also found a second specialization
that emerges in the receptive fields of V1 neurons.
Certain cortical neurons in the monkey respond well
when a stimulus moves in one direction
and poorly or not at all when the same 
stimulus is moved in the opposite direction.
This feature is called {\em direction selectivity}.
Figure~\ref{f5:dir.selective} shows the response of a
neuron in monkey area V1 to a line first
moving in one direction and then in the opposite direction.
Notice that the cell shows orientation selectivity, it only
responds well to the line in one orientation.
In addition, the cell shows direction selectivity.
When the line moves up and to the right the
cell responds well but when the same line moves down
and to the left the cell responds poorly.
Because of the low spontaneous response rate of this neuron,
which is characteristic of many cortical neurons,
we cannot tell from these measurements whether
the neuron simply fails to respond or if it is actively inhibited
by the stimulus moving in the wrong direction.
\begin{figure}
\centerline {
\psfig{figure=../06cortex/fig/dir.selective.ps,clip=,width=5.5in}
}
\caption[Direction Selectivity]{
{\em Direction selectivity} of a cortical neuron's response.
The firing pattern in response to movement in opposite directions,
indicated by the arrows, are shown.
The left hand portion of each panel shows the receptive
field location, the orientation of the line stimulus, and the 
two motion directions.
The action potentials shown on the right are the neuron's
response to motion in each of the two opposite directions.
The neuron's response depends upon the direction of motion
and the orientation of the line
(From Hubel and Wiesel, 1968).
%Receptive fields and functional architecture of monkey striate
%cortex.  J. Physiol. v. 195 p. 215-243 fig. 2.
}
\label{f5:dir.selective}
\end{figure}

The direction selective neurons are found mainly
in certain layers of the cortex and are quite rare or absent
from others.
The main layers containing direction selective
neurons are 4A, 4B, 4C$\alpha$ and layer 6 (Hawken, Parker and Lund, 1988).
These layers receive the main input from the magnocellular pathway
and send their outputs to selected brain areas.
Hence, these neurons may be part of a visual stream that is
specialized to carry information about motion.

Direction selectivity of the receptive field response
may arise from neural connections that are analogous to the
connections used underlying orientation selectivity.
A cell with a direction selective receptive field
can be built by sending the outputs of
neurons with spatially displaced receptive fields
onto a single cortical neuron
and introducing temporal delays into the path
of some of the input neurons.
The temporal delays of the signal are 
a displacement of the signal in time.
As we will review
in more detail in Chapter~\ref{chapter:Motion},
the result of a combined spatial and temporal 
displacement is to create a cortical neuron that responds
better to stimuli moving in one direction, when the delay reinforces
the signal, than to stimuli moving in the
opposite direction, when the delay works against the two signals.
This scheme for connecting neurons
is plausible; but like the mechanisms of orientation selectivity,
the precise neural wiring used to achieve direction selectivity
have not been demonstrated in primate cortical neurons.

\subsection*{Contrast Sensitivity of Cortical Cells}
Perhaps the most straightforward way to classify simple
and complex cells is based on their responses to
contrast-reversing sinusoidal patterns.
Examples of the response of a simple and a complex cell
to a contrast-reversing pattern are shown
in Figure~\ref{f5:SimpleComplex}.

Recall from Chapter~\ref{chapter:retina}
that contrast-reversing patterns are periodic in both space and time.
The stimulus used to create the neural responses shown
in Figure~\ref{f5:SimpleComplex} had a temporal period of 0.5 seconds.
Figure \ref{f5:SimpleComplex}a
shows the firing rate of a simple cell
averaged over many repetitions of the contrast reversing stimulus.
Were the simple cell perfectly linear, 
the variation in firing rate would be sinusoidal
and one period of the response would equal one period
of the stimulus.
This sinusoidal variation is impossible, 
however, because the spontaneous discharge rate of the neuron is close
to zero;
hence, the firing rate cannot fall below the spontaneous rate.
The response shown in the figure is typical of cortical simple cells
because many have a low spontaneous discharge rate.
When a signal follows only the positive part of the
sinusoid, and has a zero response to the negative part,
it is called {\em half-wave rectified}.
The response of many simple cells 
shows this half-wave rectification.

Figure \ref{f5:SimpleComplex}b shows the average response
of a complex cell during one period of the stimulus.
Unlike the simple cell,
the complex cell response does not vary at the same
frequency as the input stimulus;
the cell's response is elevated during
both phases of the flickering contrast.
This response pattern is called {\em full-wave} rectification,
and the temporal response varies
at twice the temporal frequency of the stimulus.
This nonlinear {\em frequency doubling} is typical of complex
cells.
These cells make up a large proportion of the neurons in area V1.

% Another possible good reference: (Skottun, et al., 1991).
\begin{figure}
\centerline {
\psfig{figure=../06cortex/fig/simpleComplex.ps,clip=,width=5.5in}
}
\caption[Simple and Complex Cell Responses]{
{\em The timecourse of response of cortical cells} 
to a contrast-reversing spatial frequency pattern
at a period of 0.5 seconds.
(a) The response of a simple cell
is a half-wave rectified sinusoid.
(b) The response of the complex cell is full-wave rectified.
Consequently, the temporal response
is at twice the frequency of the stimulus.
(Source: DeValois, Albrecht, Thorell, 1982). 
% reference as above, Fig 2.
}
\label{f5:SimpleComplex}
\end{figure}
DeValois, Albrecht and Thorell (1982)
measured the spatial contrast sensitivity functions of
cortical neurons.
Figure~\ref{f5:corticalTuning} shows
a sample of these measurements,
for both simple and complex cortical neurons.
The contrast sensitivity functions
of these neurons are narrower
than those of retinal ganglion cells.
Moreover, even though these measurements were made from neurons
close to one another in the cortex,
there is considerable heterogeneity in the most effective
spatial frequency of the stimulus.
This variation in spatial tuning is not true of retinal
neurons from a single class.
This may be due to a new specialization in the cortex, or
it may be that we have not yet identified the classes of
cortical neurons properly.
In either case, the different peak spatial frequencies
of the contrast sensitivity functions
raises the question of how the signals from retinal
neurons within a small patch
are recombined to form cortical neurons with
such varied spatial receptive field properties.
\begin{figure}
\centerline {
\psfig{figure=../06cortex/fig/cortical.tuning.ps,clip=,width=5.5in}
}
\caption[Spatial Frequency Tuning of V1 cells]{
{\em Spatial frequency selectivity}
of six neurons cells in area V1 of the monkey.
These responses were recorded at nearby locations within the
cortex,
yet the neurons have different spatial frequency selectivity.
(Source: DeValois et al., 1982).
%figure 4 same reference as above.
}
\label{f5:corticalTuning}
\end{figure}

Movshon, Thompson and Tolhurst (1978ab;  Tolhurst and Dean, 1987) tested
the linearity of cat simple cells.
Taking into account the low spontaneous rate and the
resulting half-wave rectification,
they found that they could
predict quantitatively a range of simple simple cell responses
from measurements of the contrast sensitivity function.
The predictions work well
for stimuli with moderate to weak contrast, that is stimuli
that evoke a response that is less than half of the maximum
response rate of the neuron.
There have not been extensive tests of linear receptive
fields in the monkey cortex, but contrast sensitivity curves
are probably adequate to predict monkey simple cell responses, too.

Figure~\ref{f5:corticalTuning} also includes contrast sensitivity
functions of nonlinear complex neurons.
Recall from our discussion in earlier chapters
that when a system is nonlinear,
its response to sinusoidal patterns
is not a fundamental measurement of the neuron's performance:
we cannot use it to predict the response to other stimuli.
For these nonlinear neurons,
the contrast sensitivity function
defines the response of the cell to an interesting
collection of stimuli.
And, these measurements may help us understand the nature
of the nonlinearity.
But, the contrast response
function of a nonlinear system
is not a complete quantitative measurement
of the cell's receptive field.

\subsection*{Contrast Normalization}
Taking into account the low spontaneous firing rate,
simple cells are approximately linear
for moderate contrast stimuli.
As one expands the stimulus range, however,
several important response properties of cortical simple
cells are nonlinear.
One deviation from linearity, called {\em contrast normalization},
can be demonstrated by measuring the contrast-response function
(cf. Figure~\ref{f4:pmContrast}).

Figure~\ref{f5:contNormData} shows
the contrast response function of a neuron in area V1 to four
different sinusoidal grating patterns.
The stimulus contrast and neuronal responses are plotted on
logarithmic axes.
The rightward displacements of the curves indicate that the
neuron is differentially sensitive to the spatial patterns
used as test stimuli.
This shift is what we expect from a simple linear system
followed by a static nonlinearity
(see the discussion in Chapter~\ref{chapter:wavelength}
near Figure~\ref{f3:phot.homogeneity}).

The entire set of data is not consistent with such a model, however,
because the response saturation level depends on
the spatial frequency of the stimulus.
Were the nonlinearity  static, then the response saturation level
would be the same no matter which stimulus we used.
Since the saturation level is stimulus-dependent,
it cannot be based on the neuron's intrinsic properties.
Rather, it must be mediated through an active process
(Albrecht and Geisler, 1991; Heeger, 1992).
This process is called {\em contrast normalization}.
\begin{figure}
\centerline{
 \psfig{figure=../06cortex/fig/contNormData.ps,clip= ,width=5.5in}
}
\caption[Contrast Normalization Data]{
{\em Contrast response functions} of a neuron in area V1.
Each curve shows the responses
measured using a different spatial frequency gratings.
The spatial frequencies of the stimuli
are shown at the right.
The neuron's sensitivity and maximum response
depend on the stimulus spatial frequency
(Source: Albrecht and Hamilton, 1982).
% Striate Cortex of Monkey and Cat:  Contrast Response FUnction, J. of
% Neurophysioloyg v. 48 no. 1 1982 D. G. Albrecht and D. B. Hamilton, 
% figure 9.  I think these are monkey because he calls them
% striate cells.  Same cell as in 7B.
}
\label{f5:contNormData}
\end{figure}

Heeger (1992) has described a model of
this process (see Figure \ref{f5:contNormModel}).
The model assumes that the neuron's response is initiated
by a linear process.
This linear signal is divided by a second
signal whose value depends on
the pooled activity of the population of cortical neurons.
This is a nonlinear term.
It is not a static nonlinearity because the divisive term
depends on the contrast of the stimulus.

This model explains the data in Figure~\ref{f5:contNormData} as follows.
First, the sensitivity of the neuron varies with the spatial
frequency of the stimulus because the initial linear receptive
field will respond better to some stimuli than others.
This causes the response to be displaced along the horizontal
axis in the log-log plot.
Second, the response saturation level depends on the
ratio of the neuron's intrinsic sensitivity to the stimulus
and the neural population's sensitivity to the stimulus.
This saturation level is set by the normalization process.
If the neuron is relatively insensitive to the stimulus compared
to the population as a whole, then the peak response of the neuron
will be suppressed by the divisive signal.
Finally, the overall shape of the response function is determined
by the nature of the static nonlinearity that follows.
\begin{figure}
\centerline{
 \psfig{figure=../06cortex/fig/contNormModel.ps,clip= ,width=5.5in}
}
\caption[Contrast normalization theory]{
{\em A model of contrast normalization} is shown.
According to this model,
each neuron's response is derved from
an initial linear encoding of the stimulus.
The linear response is divided
by a factor that depends on the
activity of the neural population.
Finally, the entire signal passes is modified by a static nonlinearity
(Source:  Heeger, 1992, 1994).
%Normalization of cell responses in cat striate cortex.  Visual
%Neuroscience, 1992 v. 9 181-197. Cambridge University Press.
% Fig. 1
%	Modification of Science paper figure.
}
\label{f5:contNormModel}
\end{figure}

What purpose does the contrast-response nonlinearity serve?
From the data in Figure~\ref{f5:contNormData},
notice that the response ratio remains
approximately constant at all stimulus contrast levels.
Without the contrast normalization process,
the neuron's response would saturate at the same level,
independent of the stimulus.
In this case, the response ratios at different contrast levels
would vary.
For example, at high contrast levels all of the neurons would
be saturated and their signals would be nondiscriminative
with respect to the input signal.
The normalization process adjusts
saturation level so that it depends on the neuron's
sensitivity;
in this way the ratio of the neuronal responses
remain constant across a wide range of contrast levels.

\subsection*{Binocular Receptive Fields}
At the input layers of the visual cortex,
signals from the two eyes are spatially segregated.
Within the superficial layers, however,
many neurons respond to light presented to either eye.
These neurons have {\em binocular} receptive fields.
Cortical area V1 is the first point in the visual pathways 
where individual neurons receive binocular input.
One might guess that these binocular neurons may play a role
in our perception of stereo depth.
What binocular information is present
that neurons might use to deduce depth?

\begin{figure}
\centerline {
\psfig{figure=../06cortex/fig/disparities.ps,clip=,width=5.5in}
}
\caption[Retinal Disparity]{
{\em Retinal disparity and the horopter} are explained.
(a) The fovea and three pairs of points 
at corresponding retinal locations are shown.
(b) When the eyes are fixated at a point F,
rays originating at corresponding points on the two retinae
and passing through the lens center
intersect on the horopter (dashed curve).
The images of points located farther (c) or
closer (d) than the horopter do not fall at corresponding
retinal locations.
}
\label{f5:disparities}
\end{figure}
First, consider the two retinae as illustrated in
Figure~\ref{f5:disparities}a.
We can label points on the two retina with respect to their
distance from the fovea.
We say that a pair of points on the two retinae fall at
corresponding locations if they are displaced from the
fovea by the same amount.
Otherwise, the two points fall at non-corresponding positions.

Now, suppose that the two eyes are positioned so that
a point F casts an image on the two foveae.
By definition, then, the images of the  point F fall
on corresponding retinal locations.
By tracing a ray from the corresponding retinal positions
back into space,
we can find the points in space whose images 
are cast on corresponding
retinal positions (Figure~\ref{f5:disparities}b).
These points sweeps out an
arc about the viewer that is called the {\em horopter}.

The image of a point closer or further
than the horopter will fall
on non-corresponding retinal positions.
The difference between the image locations and
the corresponding locations is called the
{\em retinal disparity}.
Because the main separation
between the two eyes is horizontal, the
retinal disparities are mainly in the horizontal direction as well.
The horopter is the set of points whose images have zero retinal disparity.

Figures~\ref{f5:disparities}cd show
two examples in which image points
fall on noncorresponding retinal points.
Figure~\ref{f5:disparities}c shows an example
when both images fall on the nasal side of the foveae,
and Figure~\ref{f5:disparities}d shows an example
when both images fall on the temporal side of the fovea.
These panels show that the size and nature of the horizontal
retinal disparity varies with the distance
from the visual horopter.
Hence, the horizontal retinal disparity is a
binocular clue
for estimating the distance to an image point\footnote{
You can demonstrate the relative shift in retinal positions
to yourself as follows.
Focus on a nearby object, say your finger placed in front of your nose.
Then, alternately look through one eye and then the other.
Although your finger remains in the fovea, the relative
positions of points nearer or further than your finger
will change as you look through each eye in turn.}.

Do binocular neurons represent stereo depth information
by measuring horizontal disparity?
There are two types of experimental measurements we can make
to answer this question.
First, we can measure the receptive fields of 
individual binocular neurons.
If retinal disparity is used to estimate depth,
then the receptive fields of the binocular neurons should show
some selectivity for horizontal disparity.
Second, we can look at the properties of the population
of binocular neurons.
While no single neuron alone can code depth information,
the population of binocular
neurons should include enough information
to permit the population to estimate image depth.

A complete characterization of binocular receptive fields
requires many measurements.
First, one would like to measure the spatial receptive fields
of the neuron when stimulated by each eye alone.
These are called the {\em monocular} receptive fields of the
binocular neuron.
Then, we should characterize how the binocular neuron responds
to simultaneous stimulation of the two eyes.
In practice there have been very few
complete measurements of binocular neurons'
receptive fields.
The vast majority of investigations have 
been limited to localization of the monocular
receptive field centers that are then used to
derive the retinal disparities between
the monocular field centers.

Given the variability inherent in biological systems,
the two monocular receptive fields will not be in perfect register.
We would like to decide whether the observed
horizontal disparities are purposeful,
or whether they are due to unavoidable random variation.
To answer this question several groups have
measured both the horizontal and the vertical disparities
of binocular neurons in the cat cortex (Barlow et al., 1967;
Joshua and Bishop, 1970; von der Heydt, 1978).
The histograms in Figure~\ref{f5:neuralDisparity}a show
the initial measurements from
Barlow et al. (1967).
They observed more variability in the horizontal
disparity than vertical disparity, and they
concluded that the horizontal variation was purposeful
and used for processing depth.
\begin{figure}
\centerline{
 \psfig{figure=../06cortex/fig/neuralDisparity.ps,clip= ,width=5.5in}
}
\caption[Neural disparities in cat]{
{\em The horizontal and vertical disparities of binocular neurons} 
in the cat visual cortex are shown.
(a) Histograms of the horizontal and vertical disparities of
binocular neurons in cat cortex (Source: Barlow  et al., 1967).
(b) A scatter diagram of the vertical and horizontal disparities
of cells in cat cortex with
receptive fields located
within 4 degrees of the cat's best region
of visual acuity.
(Source: Bishop, 1973)
% Handbook of sensory physiology, Berlin, 1973, 
% Springer-Verlag, vol. 7/3A, p. 268
% This should be the Barlow figure.
}
\label{f5:neuralDisparity}
\end{figure}
Joshua and Bishop (1970) and van der Heydt (1978)
saw no difference in the range of disparities
in the horizontal and vertical directions.
A scatter plot of the retinal disparities observed
by Joshua and Bishop (1970) is shown in Figure~\ref{f5:neuralDisparity}b.
While these data do not show any systematic difference
between the horizontal and vertical disparity distributions,
these authors do not dispute Barlow et al.'s hypothesis that
variations in the horizontal disparity are used for
stereo depth detection\footnote{
A frequently suggested alternative 
is that these disparity cues
serve to converge the two eyes.
Since the same cues are used to converge the eyes and
estimate depth, this alternative hypothesis is virtually
impossible to rule out.}.

For the moment, let's accept the premise that the
variation in horizontal disparity of these binocular neurons
is a neural basis for stereo depth.
How might we design the binocular response properties of these
neurons to estimate depth?

One possibility is to create a collection
of neurons that each responds to only a single disparity.
One might estimate the local disparity by identifying
the neuron with the largest response.
An alternative possibility,
suggested by Richards (1971),
is that one might measure
disparity by creating a few {\em pools} of neurons with
coarse disparity tuning.
One pool might consist of neurons
that respond when an object feature
is beyond the horopter,
and a second pool consists of neurons
that respond when the feature is in front of it.
The third pool might respond only when the feature is close
to the horopter.
To estimate depth, one would compare the relative
responses in the three neural pools.

Some support for Richards' hypothesis comes from
measurements of individual neurons
in areas V1 and the adjacent area V2 of a monkey brain.
Poggio and Fisher (1981; see also Ferster, 1981) measured
how well individual neurons
respond to stimuli with different amounts of disparity.
They used experimental stimuli consisting
of bar patterns whose width
and velocity were set to generate a strong
response from the individual neuron.
The experimenters varied the retinal disparity between
the two bars presented to the two eyes.
They plotted the binocular neuron's response 
to the moving bars as a function of their retinal disparity.
The curves in Figure~\ref{f5:binoc}, plotting
response as a function of retinal disparity,
are called {\em disparity tuning} curves.

\begin{figure}
\centerline{
\psfig{figure=../06cortex/fig/binoc.ps,clip= ,width=5.5in}
}
\caption[Binocular Neurons and the Pool Hypothesis]{
{\em Disparity tuning curves} of binocular neurons in areas V1 and V2
in monkey.  Each panel plots the response of a different neuron to
moving bar patterns.  The independent variable is the retinal
disparity of the stimulus.
(a) and (b) show the responses of neurons
that respond best to stimuli with near zero disparity, that is near
the horopter.  Responses of a neuron that responds best to stimuli
with positive disparity (c) and a neuron with negative disparity (d)
are also shown.  The curves in represent data
measured using binocular stimulation.  
(Source: Poggio and Talbot, 1981).
% G. F. Poggio and W. H. Talbot,
% J. Physiol.p. 469-492 Mechanisms of static and dynamic 
% stereopsis in foveal cortex of the rhesus monkey,
% v, 315, 1981
}
\label{f5:binoc}
\end{figure}
Poggio and Talbot (1981) found that the disparity tuning curves
could be grouped into a small number of categories.
Typical tuning curves
from each of these categories are are illustrated in
the separate panels of Figure \ref{f5:binoc}.
The two neurons illustrated
in the left panels respond to disparities
near the fixation plane;
for these neurons stimuli near the
horopter stimulate or inhibit the cell.
The two panels on the right
illustrate neurons with opponent tuning.
One neuron is excited by a bar whose
disparity places the object beyond
the horopter and the neuron is inhibited by bars in front of the horopter.
The second neuron shows approximately the complementary
excitation pattern.
The neuron is excited by objects nearer than the 
horopter and inhibited by objects further.
Poggio and his colleagues view their measurements in monkey
as support for Richards' hypothesis that binocular
depth is coded based on the response of neurons
organized in disparity pools (also see Ferster, 1981).

\begin{figure}
\centerline{
 \psfig{figure=../06cortex/fig/ohzawa.ps,clip= ,height=4.0in,width=1.75in}
}
\caption[Monocular Spatial Receptive Fields of Cat Binocular Neurons]{
{\em Monocular spatial receptive fields} of 
two binocular neurons in cat cortex.
(a) and (b) show examples of left (L) and right (R)
monocular receptive fields whose centers are displaced
horizontally and thus have non-zero retinal disparity.
In addition to the disparity, the left
and right monocular spatial receptive fields differ.
(Source: Freeman and Ohzawa, 1990).
}
\label{f5:ohzawa}
\end{figure}
We have been paying attention mainly to the retinal disparity
of the binocular neurons.
But, disparity tuning is only one measure of the receptive field
properties of these neurons.
In addition, the receptive fields must have spatial,
temporal and chromatic selectivities.
To fully understand the responses of these neurons we must make
some progress in measuring all of these properties.

To obtain a more complete description of binocular neurons,
Freeman and Ohzawa (1990;  DeAngelis et al., 1991) studied the monocular
spatial receptive fields of cat binocular neurons.
They found that the spatial receptive fields
measured in the two eyes can be quite different.
Figure~\ref{f5:ohzawa} shows
an example of the differences they observed
between the spatial monocular receptive fields.
The left eye spatial receptive field and the right eye
monocular field are displaced relative to one another.
If we only concern ourselves with disparity, we will
report that this cell's receptive
field has significant horizontal disparity.
But, notice that the spatial receptive fields are different
from one another.
The spatial receptive field in the left eye is a mirror-reversal
of the field in the right eye.

Freeman and Ohzawa suggest that these  different
receptive spatial monocular receptive fields are important to the
way in which stereo depth is estimated by the nervous system.
They hypothesize that stereo depth depends on having neurons
with different monocular spatial receptive fields.
Perhaps most important, however, their measurements
reminds us that to understand the biological computation
of stereopsis, we must study more than just
the center position of the monocular receptive fields.
