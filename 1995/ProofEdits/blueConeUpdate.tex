% Documentstyle included in preamble.tex
%
\input{/u/brian/lib/tex/standardBook}
\begin{document}

\def\sfrac#1#2{{\textstyle \frac{#1}{#2}}}   
\def\mat#1{{\bf #1}}

%
%	This resides in /usr/share/src/Tex/lib/inputs
%
\input{psfig}

% \newcommand{\psfigpath}{/home/brian/images/general}
%
% 	I think this sets it up for draft mode rather than actually
% 	including the stuff
%\psdraft


%
%	All the math includes
%
%	From Chapter on Image Formation
%
%

%%%%%%%%%%%%%%%%%%%%%%%%%%%%%%%%
%
%
%	General vector and matrix notation
%
\newcommand{\vect}[1]{\mbox{${\bf #1}$}}
\newcommand{\vecti}[2]{\mbox{${#1}_{#2}$}}
\newcommand{\vectij}[3]{\mbox{${#1}_{#2,#3}$}}
\newcommand{\matr}[1]{\mbox{${\bf #1}$}}

\newcommand{\lin}{\mbox{{ \boldmath $L$ } }}
\newcommand{\twopN}{\mbox{${\frac{2 pi}{N}}$}}
\newcommand{\sumoneNn} {\sum_{n = 1}^{N}}
\newcommand{\sumoneNi} {\sum_{i = 1}^{N}}
\newcommand{\sumoneNj} {\sum_{j = 1}^{N}}
\newcommand{\sumoneNk} {\sum_{k = 1}^{N}}
\newcommand{\sumoneNf} {\sum_{f = 1}^{N}}
\newcommand{\sumoneNfx}{\sum_{fx = 1}^{N}}
\newcommand{\sumoneNfy}{\sum_{fy = 1}^{N}}
\newcommand{\sumzeroNn}{\sum_{n = 0}^{N - 1}}
\newcommand{\sumzeroNi}{\sum_{i = 0}^{N - 1}}
\newcommand{\sumzeroNj}{\sum_{j = 0}^{N - 1}}
\newcommand{\sumzeroNk}{{\sum_{k = 0}^{N - 1}}}
\newcommand{\sumzeroNf}{\sum_{f = 0}^{N - 1}}
\newcommand{\sumzeroNfx}{\sum_{fx = 0}^{N - 1}}
\newcommand{\sumzeroNfy}{\sum_{fy = 0}^{N - 1}}
\newcommand{\sumoverNi} {\sum_{i = \langle N \rangle}}
\newcommand{\sumoverNj} {\sum_{j = \langle N \rangle}}
\newcommand{\sumoverNk} {\sum_{k = \langle N \rangle}}

\newcommand{\Dzero}{\mbox{$(1,0,0, \ldots ,0)^t$}}
\newcommand{\Done}{\mbox{$(0,1,0, \ldots ,0)^t$}}
\newcommand{\DN}{\mbox{$(0,0,...,0,1)^t$}}
\newcommand{\Di}{\mbox{$(0,..,1,...0)^t$}}

%	Starting with Chapter on Optics
%
\newcommand{\lspread}{\mbox {$\bf l$} }
\newcommand{\lspreadi}[1]{\mbox{$l_{#1}$}}
\newcommand{\lspreadhat}{\mbox {$\hat{\bf l}$}}
\newcommand{\lspreadhati}[1]{\mbox {$\hat{l}_{#1}$}}
\newcommand{\Resmat}{\mbox {$\bf R$} }
\newcommand{\resvec}{\mbox {$\bf r$} }
\newcommand{\resveci}[1]{\mbox{$r_{#1}$}}
\newcommand{\resvechat}{\mbox {\bf $\hat{r}$}}
\newcommand{\resvechati}[1]{\mbox {$\hat{r}_{#1}$}}
\newcommand{\pixmat}{\mbox {$\bf P$} }
\newcommand{\pixvec}{\mbox {$\bf p$} }
\newcommand{\pixveci}[1]{\mbox{$p_{#1}$}}
\newcommand{\pixvechat}{\mbox {$\bf \hat{p}$}}
\newcommand{\pixvechati}[1]{\mbox{$\hat{p}_{#1}$}}
\newcommand{\Optics}{\mbox{$\bf O$}}
\newcommand{\pfiN}{\mbox{$\frac {2 \pi f i}{ N }$}}
\newcommand{\SfNi}{\mbox{$\sin  ( \frac {2 \pi f i}{ N } )$}}
\newcommand{\SfnNi}{\mbox{$\sin  ( - \frac {2 \pi f i}{ N } )$}}
\newcommand{\SfiNk}{\mbox{$\sin ( \frac {2 \pi f_i k}{ N } )$}}
\newcommand{\SfjNk}{\mbox{$\sin ( \frac{2 \pi f_j k}{ N } )$}}
\newcommand{\CfNi}{\mbox{$\cos (  \frac{2 \pi f i}{ N } )$}}
\newcommand{\CfiNk}{\mbox{$\cos ( \frac{2 \pi f_i k}{ N })$}}
\newcommand{\CfjNk}{\mbox{$\cos ( \frac{2 \pi f_j k}{ N } )$}}
\newcommand{\SfNj}{\mbox{$\sin (  \frac{2 \pi f j}{ N } )$}}
\newcommand{\CfNj}{\mbox{$\cos (  \frac{2 \pi f j}{ N })$}}
\newcommand{\SfNk}{\mbox{$\sin (  \frac{2 \pi f k}{ N })$}}
\newcommand{\SfnNk}{\mbox{$\sin ( - \frac{2 \pi f k}{ N })$}}
\newcommand{\CfNk}{\mbox{$\cos (  \frac{2 \pi f k}{ N })$}}
\newcommand{\SfNij}{\mbox{$\sin ( \frac{2 \pi f (i + j)}{ N } )$}}
\newcommand{\SfoneNk}{\mbox{$\sin ( \frac{2 \pi f_1 k}{ N } )$}}
\newcommand{\SftwoNk}{\mbox{$\sin ( \frac{2 \pi f_2 k}{ N } )$}}
\newcommand{\CftwoNk}{\mbox{$\cos ( \frac{2 \pi f_2 k}{ N } )$}}
\newcommand{\matCos}{\mbox{$ \bf C$}}
\newcommand{\matSin}{\mbox{$ \bf S$}}

%	From chapter on color
%

\newcommand{\Red}{\mbox{L}}
\newcommand{\Green}{\mbox{M}}
\newcommand{\Blue}{\mbox{S}}
\newcommand{\redw}{\Red ( \lambda )}
\newcommand{\greenw}{\Green ( \lambda )}
\newcommand{\bluew}{\Blue ( \lambda )}
\newcommand{\evec}{\mbox{$\bf e$}}
\newcommand{\eveci}[1]{\mbox{$e_{#1}$}}
\newcommand{\er}{e_r}
\newcommand{\eg}{e_g}
\newcommand{\eb}{e_b}
\newcommand{\test}{\mbox{${\bf t}$}}
\newcommand{\testi}[1]{\mbox{$t_{#1}$}}
\newcommand{\testw}{t ( \lambda )}
\newcommand{\lms}{\mbox{$\bf r$}}
\newcommand{\Iodopsin}{\mbox{$\bf B$}}
\newcommand{\Rhodopsin}{\mbox{$\bf A$}}
\newcommand{\Rhodopsinw}{\mbox{$\bf A(\lambda)$}}
\newcommand{\Scotopic}{\mbox{$\bf R$}}
\newcommand{\Scotopicw}{\mbox{$R ( \lambda )$}}
\newcommand{\Scotopici}[1]{R_{#1}}
\newcommand{\primaryint}{\mbox{$\bf e$}}
\newcommand{\primaryinti}[1]{\mbox{$e_{#1}$}}
\newcommand{\primary}{\mbox{$\bf p$}}
\newcommand{\primaryi}[1]{\mbox{${\bf p}_{#1}$}}
\newcommand{\Primarymat}{\mbox{$\bf P$}}
\newcommand{\Photopic}{\mbox{\bf C}}
\newcommand{\Monitor}{\mbox{$ \bf M $}}
\newcommand{\Calibration}{\mbox{$\matr{H}$}}
\newcommand{\monitori}[1]{\mbox{$\bf m_{#1}$}}
\newcommand{\Sensor}{\mbox{\bf S}}
\newcommand{\current}{\mbox{\bf i}}
\newcommand{\currentt}{\mbox{$i ( t )$} }
\newcommand{\nl}{\mbox{$n_\lambda$}}

%
%	From chapter  on physiology
%
\newcommand{\Pbeta}{P_\beta}
\newcommand{ \amp} { Amplitude }
\newcommand{ \sumxx} { \sum_{x = 0} to {x = N} }
\newcommand{ \sumtt} { \sum_{t = 0} to {t = N} }
\newcommand{ \Aftx} { A ( ft , x )}
\newcommand{ \BftxC} { {  W ( ft , C(x) ) } }
\newcommand{ \dt} { { delta t } }
\newcommand{ \fx} {f_{x}}
\newcommand{ \ft} {f_{t}}
\newcommand{ \phix} {p_{x} }
\newcommand{ \phit} {p_{t} }
\newcommand{ \contx} { a_{x} }
\newcommand{ \contt} { a_{t} }
\newcommand{ \intensityi}[1]{\mbox{$i_{#1}$}}
\newcommand{\mean}{\mbox{$m$}}
\newcommand{\pspread}{\mbox {$\bf p$} }
\newcommand{\pspreadi}[1]{\mbox{$p_{#1}$}}
\newcommand{\pspreadhat}{\mbox {\bf $\hat{p}$}}
\newcommand{\pspreadhati}[1]{\mbox {$\hat{p}_{#1}$}}
\newcommand{ \RF} {\mbox{ \bf R}}	%Receptive field
\newcommand{ \linex} {\delta(x)}
\newcommand{ \responsemat} {\bf F}
\newcommand{ \response} {\bf f}
\newcommand{ \responsei} {\mbox{$f_{i}$} }
\newcommand{ \responset} {f ( t ) }
\newcommand{ \contrast}{\mbox{$a$}}
\newcommand{ \contrasti}[1]{\mbox{$a_{#1}$}}
\newcommand{ \contrastvec}{\mbox{$\bf a$}}
\newcommand{ \contrastmat}{\mbox{$\bf A$}}
\newcommand{ \onecontrastmat} {\bf U}
\newcommand{ \LW} {\bf L}
\newcommand{ \As} {A_{s}}
\newcommand{ \Bs} {B_{s}}
\newcommand{ \RFs} {RF_{s}}
\newcommand{ \Ac} {A_{c}}
\newcommand{ \deltai} {(0,..,1,...0)^{t}}

%
%	From chapter on Pattern Vision
%
\newcommand{\svec}{\mbox{$\bf s$}}
\newcommand{\wvec} { \mbox{$\bf w$}}
\newcommand{\stim} { \mbox{$\bf s$} }
\newcommand{\stimi}[1]{ \mbox{$\bf s{_#1}$} }
\newcommand{\neural}{\mbox{$\bf n$}}		%Neural image vector
\newcommand{\neuralx}[1]{\mbox{$n_{#1}$}}
\newcommand{\length}{\mbox{$d$}}
\newcommand{\lengthi}[1]{\mbox{$d_{#1}$}}
\newcommand{\sinlengthi}[1]{\mbox{$a_{#1}$}}

%
%	From chapter on multiresolution

\newcommand{\gest}[1]{\mbox{$\hat{g}_{#1}$}}

%	From the chapter on Color Appearance
%
\newcommand{\recSens}[1]{\mbox{$R_{#1}$}}
\newcommand{\recResp}{\mbox{$\bf r$}}
\newcommand{\recRespi}[1]{\mbox{$r_{#1}$}}
\newcommand{\sensorMat}{\mbox{$\bf S$}}

\newcommand{\colsig}[1]{\mbox{$c(#1)$}}
\newcommand{\colsigi}[1]{\mbox{$c_{#1}$}}

\newcommand{\ill}[1]{\mbox{$e(#1)$}}
\newcommand{\illhat}[1]{\mbox{$\hat{e}(#1)$}}
\newcommand{\illi}[2]{\mbox{$e_{#1}(#2)$}}
\newcommand{\illvec}{\mbox{$\bf e$}}

\newcommand{\illveci}[1]{\mbox{$\bf e_{#1}$}}
\newcommand{\illBasis}{\mbox{$\bf B_e$}}
\newcommand{\illBasisi}[2]{\mbox{$E_{#1}(#2)$}}
\newcommand{\illCoef}{\mbox{${\bf \omega}$}}
\newcommand{\illCoefi}[1]{\mbox{$\omega_{#1}$}}
\newcommand{\illMat}[1]{\mbox{$\Lambda_{#1}$}}
\newcommand{\illMatinv}[1]{\mbox{$\Lambda^{-1}_{#1}$}}

\newcommand{\surf}[1]{\mbox{$s(#1)$}}
\newcommand{\surfvec}{\mbox{$\bf s$}}
\newcommand{\surfveci}[1]{\mbox{$s_{#1}$}}
\newcommand{\surCoef}{\mbox{${\bf \sigma}$}}
\newcommand{\surCoefi}[1]{\mbox{$\sigma_{#1}$}}
\newcommand{\surBasis}{\mbox{$\bf B_s$}}
\newcommand{\surBasisi}[2]{\mbox{$S_{#1}(#2)$}}


%	Motion chapter
%
\newcommand{\parder}[2]{\mbox{$\frac{\partial #1}{\partial #2}$}}

%	Appendix A:  Shift-invariance
%
\newcommand{\Cyclic}{\mbox{$\bf C$}}
\newcommand{\Cyclichat}{\mbox{ $\hat{{\bf C}}$}}
\newcommand{\Cyclichati}[1]{\mbox{ $\hat{{C_{#1}}}$}}

%	Appendix B:  Color calibration
%

%	Appendix C:  Classification
%
\newcommand{\data}{ \mbox{${\bf d}$}}
\newcommand{\stimA}{\mbox{$A$}}
\newcommand{\stimB}{\mbox{$B$}}
\newcommand{\muAi}{\mbox{$\mu_{A,i}$}}
\newcommand{\muBi}{\mbox{$\mu_{B,i}$}}


%
%	From chapter on Tools
%
\newcommand{\dvec}{ \bf d }
\newcommand{\sd}{\sigma}
\newcommand{\muA}{\mu_{A} }
\newcommand{\sdA}{\sigma_{A} }
\newcommand{\muB}{\mu_{B}}
\newcommand{\sdB}{\sigma_{B}}
\newcommand{\mui}{\mu_{i} }
\newcommand{\sdi}{\sigma_{i} }

%	Appendix D:  Signal Estimation
%

%	Appendix E:  Motion flow
%
\newcommand{\motFlow}[2]{\mbox{${\bf m}(#1,#2)$}}

\begin{document}
\doublesp

\section*{Proofs, near Fig of Blue Cone Mosaic, Chapter 3}


\underline{Figure Caption}

{\bf A biological estimate of the $\Blue$ cone spatial mosaic} in the
macaque retina.  This image is a cross-section of the retina at the
cone inner segment layer.  The retina has been stained using a procion yellow
dye.  Cone inner segments that absorbed the procion yellow dye
strongly appear as lightly colored large spots; the dark spots show
the positions of cone inner segments that did not absorb the stain
strongly.  The small circles in between the cones show the positions
of the rods.  After DeMonasterio et al. (1985). (This image
provided by S. Schein).

% After DeMonasterio, McCrane, Newlander, Schein Density profile of
% blue-sensitive cones along the horizontal meridian of macaque
% retina.  IOVS v. 26 p. 289-302, 1985.


\underline{Nearby Text}


DeMonasterio et al. (1985) discovered that when the procion yellow dye
is applied to the retina, it stains only a small subset of the
photoreceptors inner segments completely.  Figure 3.7 shows a spatial
mosaic of these stained cones in a tangential section of the retina at
the inner segment level just outside the fovea.  The inner segments
that take up this dye fully appear as light disks; the dark disks are
also cone inner segments; the small disks in between are rods.  By
studying many such images across the retina, De Monasterio et
al. found that these cones are are absent from the central fovea, they
have their peak density 1 deg from the central fovea, and they are
spaced widely compared to the other cones.  These factors are
consistent with the psychophysical observations concerning the
positions of the $\Blue$ cones, and thus it seems likely that the
stained inner segments are the $\Blue$ cones.  

\newpage

Why do the $\Blue$ cones stain more strongly than the other cone
types?  The $\Blue$ cones are generally more susceptible to disease
than the other two cone types.  Hence, it is possible that when they
react to the procion yellow dye, their membranes break apart more
readily than the $\Red$ and $\Green$ cone membranes.  Once the inner
segment membranes come apart, the $\Blue$ cones stain more completely.

\end{document}
