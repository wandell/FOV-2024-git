The space-time oriented linear filters respond
strongly to motion in of a particular
velocity and direction, and less
strongly to other velocities and direction.
The basic idea of designing these filters is to use
the responses from a large collection of filters to
form an estimate of the motion flow field.
We need to use
filter responses at each point
in the visual field to decide
on the local motion direction.
We need a rule to specify how the filter outputs
should be interpreted to yield a motion flow vector.

The consensus on
the spatial filtering required for motion
does not extend
to the methods of interpreting the filter responses
(Adelson and Bergen, 1985; van Santen and Sperling, 1985;
Watson and Ahumada, 1985).
I will describe one
operation for combining motion sensors,
called {\em motion energy},
because it has achieved some popularity and is
used in the design of psychophysical stimuli.
It seems likely to me, however, that over the next few
years, the motion-energycomputation
will be replaced by more sophisticated ideas
using Bayesian Estimation.

The motion energy computation begins with
an array of linear filters that are designed
to complement one another in a special way:
the array of filters come in {\em quadrature-pairs}.
Two space-time filters are a quadrature-pair
when the spatial and temporal frequency responses of the
two filters are shifted by one-quarter of a cycle, $\pi / 2$.
Shifting a sinusoid one quarter cycle
turns it into a cosinusoid.
Hence, sinusoids and cosinusoids are quadrature pairs.
If a filter is purely an even function ($f(x) = f(-x))$,
it can be written as
the weighted sum of cosinusoids.
Its quadrature pair will be
purely odd ($f(x) = -f(-x)$), and hence the sum of sinusoids.
The use of quadrature-pair filters for motion energy
was suggested by Watson and Ahumada, who use it for the
basis of a different scheme,
and also used by Adelson and Bergen in their
motion energy calculation.
Quadrature-pair filters are important in linear transform theory
generally, and
there exists a linear transformation, called the {\em Hilbert
Transform} that takes a filter as input and computes a companion
filter, in quadrature relationship, as output.
Figure \ref{f9:motionEnergy} illustrates a pair of
one-dimensional spatial filters in quadrature.

\begin{figure}
\centerline{
%  \psfig{figure=../09motion/fig/motionEnergy.ps,clip= ,width=3.5in,height=3.5in}
}
\caption[Motion Energy Computation]{
The motion-energy computation is a popular method for
combining linear filter outputs.
The method involves squaring and summing the responses
of quadrature-pairs of
linear filters to yield an estimate of the local direction
of motion, that is, the motion flow field.
}
\label{f9:motionEnergy}
\end{figure}

Each of the filters in the
quadrature-pair respond with a time-varying signal.
We form the motion energy of the quadrature pair
by squaring the individual signals and summing them.



Motion energy does change a time-varying signal
into a constant signal when (a)  the signal is
a sinusoid, or (b) the receptive field is sharply
tuned in temporal frequency.

\comment{Fahle and Poggio, Watson and Ahumada, 1983; Heeger,
1987,1988, Grzywacz and Yuille, 1990

How do we identify motion?  From the figure, we see
   that motion defines an oriented linear path in space-time
   diagrams.  The slope of the orientation defines the
   velocity of the motion

Notice, also, that if the space time image is non-zero 
   only on a line (or a plane) then in the representation
   in the transform domain will also be a line or a plane
   becaue the transform domain is a linear transformation
   away.

It follows that the spatio-temporal frequency
    energy must be in a plane.

One way to think about the proble is that we are trying
   to identify locally linear segments in the space-time
   description of the stimulus.

We can try to do this by using the matched filter theorem.
    Create filters that are tuned to specific space-time
    patterns.  

How do we create filters?  Sum of separable filters to
   yield non-separable filters.  Not too deep, or very deep,
   depending on your point of view.

But, since there is no single response we are looking for,
    we create a set of filters and look at the collection of
    responses;  the assumption is that the motion is most like
    the response of the strongest response.

What is wrong with this solution?  Well, remember that the
   filters are linear.  So, when we move a sinusoid through their
   receptive field we also measure a sinusoidal response.  The
   amplitude of the sinusoidal response depends on a number of different
   parameters including the velocity, contrast, spatial frequency
   and so forth of the stimulus.
}

Adelson and Bergen suggest that we use a pair of filters,
in quadrature (sin/cos) spatial phase relationship,
to track motion in an area.
The outputs of these filters are in temporal quadrature, so
squaring and summing the outputs, when the input
is a sinusoid, yields a constant response.

\[
A^2(fx) sin^2(2pift) + A^2(fx) cos^2(2pift) = A^2(fx)
\]

In this form, for a sinusoid, the constant level of the ``energy'' detector is
proportional to the spatial sensitivity of
the detector's spatial response to the stimulus.

For general patterns there is temporal variation.
A moving stimulus, at velocity v deg/sec, has a relationship
in terms of fx (c/deg), ft (cycles/sec) of

\[
ft  / fx = v  (c/sec / c/deg = deg/sec)
\]

\[
  ft = v fx
\]

\[
(sum_fx A(fx) sin(v fx))^2 + (sum_{fx} A(fx) cos(v fx))^2
\]

which doesn't factor into anything simple.
The only way this can be avoided is if the spatial sensitivity
term, A(fx) is very narrow.

If it is narrow, and there are many,
then we can use the response at each level to
tell us about the spatial pattern.

Coarse to fine tracking of motion


