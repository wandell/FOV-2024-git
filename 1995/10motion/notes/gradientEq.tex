\documentstyle[12pt]{article}
% Documentstyle included in preamble.tex
%
\input{/u/brian/lib/tex/standardBook}
\begin{document}

\def\sfrac#1#2{{\textstyle \frac{#1}{#2}}}   
\def\mat#1{{\bf #1}}

%
%	This resides in /usr/share/src/Tex/lib/inputs
%
\input{psfig}

% \newcommand{\psfigpath}{/home/brian/images/general}
%
% 	I think this sets it up for draft mode rather than actually
% 	including the stuff
%\psdraft


\begin{document}

\section{Notes on Gradient Calculations for Motion}

The stimulus intensity at $x,y$ at moment $t$ is $f(x,y,t)$.

The local change in the stimulus due to motion is approximated as
a shift in the image intensity by an amount that depends on velocity.

\begin{equation}
\label{eq:energyConservation}
f(x,y,t) = f(x + v_x, y + v_y, t + 1 ).
\end{equation}

where $v_x$ is the velocity in the x direction and
$v_y$ is the velocity in the y-direction.

What are the approximations here?  Who says that the local
image is a shifted copy over time?  If the image is created
by moving an object in three space, then the assumption can
break, say by introducing a scaling factor.  Or, if the
object moves behind/infront of another object.

We call $\hat{v} = (v_x,v_y)$ the velocity vector.
We want to estimate this vector.

Calculate the Taylor series expansion around $(x,y,t)$

\begin{equation}
f(x+v_x,y+v_y,t+1) \approx f(x,y,t) 
    + v_x f_x(x,y,t) + v_y f_y(x,y,t) + f_t(x,y,t)
\end{equation}

Putting the equations together Equation \ref{eq:energyConservation}
and the Taylor series approximation yields the approximation

\begin{equation}
\label{eq:gradientConstraint}
v_x f_x(x,y,t) + v_y f_y(x,y,t) + f_t(x,y,t) = 0
\end{equation}

which is often called the {\em gradient constraint} for
image motion.  

The idea behind gradients is that you can
estimate properties of the image,
such as the image derivative over time, w.r.t. to the variables 
$x, y, and t$,
at location $(x,y,t)$.  
In other words, all of the partial
derivatives, like the $f_x$ terms,
come from directly measuring the image.

The velocity terms are supposed to come by estimating the
image contents (derivatives).

The gradient constraint equation describes what 
happens at one point in the image.  

There are many points in the image.  
Use them all and
minimize sum of squares over many points
in the gradient constraint to obtain estimates of $\hat{v}$.

For example, if you have some reason to believe that a local
collection of points, $x_i , y_i$ have moved together, then
you should be able to solve for the velocity vector
by minimizing the equation

\begin{equation}
\sum_{i} v_x f_x(x_i,y_i,t) + v_y f_y(x_i,y_i,t) + f_t(x_i,y_i,t) = 0
\end{equation}

Taking this equation at many locations yields the
matrix problem.

\[
0 = M v + b, ~~\mbox{or} ~~ -b = Mv
\]

we pull out $(v_x,v_y)$ and solve $M v + b$
where the $b$ is the vector of temporal derivative
measurements, and the $M$ matrix contains the spatial derivatives.

\end{document}
