  First, motion perception plays a role in our
ability to identify distinct objects within an image, a process that
is called {\em image segmentation}.  When a collection of image points
share a common motion, we can use their motion to identify the points
as part of a single object.  For example, if an image consists of a
random collection of dots, moving some of the dots together produces a
strong sense of image segmentation and the dots seem to be part of a
single object.  The motion of the dots can serve as a powerful enough
stimulus to evoke the percept of a three dimensional transparent
objects.
