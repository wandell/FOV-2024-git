
Second, we analyzed the Baylor et al. experiment
using a graphical method illustrated in Figure \ref{f3:phot.homogeneity}.
In that Figure we plotted the peak
photocurrent response as a function of the logarithm of
the test light intensity.
Plotting the output in terms of the
logarithm of the input intensity simplifies matters
for a subtle reason.
Consider how we can change the representation
when we plot the logarithm of test intensity.
\begin{eqnarray}
\mbox{Photocurrent} & = & F( a \Rhodopsin \primary ) \nonumber \\
  & = & G( \log( a \Rhodopsin \primary)) \nonumber \\
  & = & G( \log( a ) + \log ( \Rhodospin \primary) ) \nonumber \\
\mbox{where}~~~~G(x) & = & F( e^{x})~~.
\end{eqnarray}
We had to substitute the function $G() =  F( e^{x} )$ so that we
can express the output in terms of the the logarithm of the input intensity.
But, since the function $F()$ is an arbitrary static nonlinearity,
there is no real penalty for converting to $G()$, yet another
arbitrary static nonlinearity.

The advantage is that
in this representation, 
we can see that the photocurrent depends
on the sum of two terms.
One is the logarithm of the test intensity, which
we systematically vary during the experiment.
The second term is
the logarithm of the sensitivity to the
test light, $\log ( \LongWavelength^t \testw)$.
This term is fixed when we choose the test light.
Hence, as we make this plot for different
test lights, the curves we sweep out should be the same
except for a displacement along the horizontal
axis.
This displacement reflects
the differential sensitivity to the test lights.
By measuring with monochromatic test lights,
we can use the size of the displacement
to estimate the
spectral sensitivity $\LongWavelength$.
Baylor, Nunn and Schnapf used this method to measure
the cone photoreceptor spectral responsivity,
despite the nonlinear data.
