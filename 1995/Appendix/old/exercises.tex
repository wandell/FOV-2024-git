\item  The performance of a shift invariant linear system can
be described by the convolution formula.
In the text we derived the convolution formula for a discrete
system with cyclic functions, $h$ and $p$, of period $N$:

\[
r_i = \sum_{j = \langle N \rangle } h_{ i - j } p_j .
\]

 \begin{enumerate}

 \item Re-write the convolution formula 
substituting the variable $k = i - j$ for the summation index,
in place of $j$.

 \item Using the new convolution formula,
compute the result when the input vector is
$p_i = e^{ i }$.
(Remember that $e^{ i + j} = e^{i} e^{j}$.
Compare the computed response, $r_i$, and the input vector, $p_i$.


 \item  Recall the trigonometric identity
\[
\sin ( i + j) = \sin (i) \cos (j) + \cos (i) \sin (j).
\]
Use the convolution formula you just derived to compute the response
of the system when $p_i = \sin (i)$.
Make sure to group the terms in the result so that the 
constants, those terms that are independent of the index $i$, 
are grouped together.

 \ee

\item Answer the following general questions about linear systems.
 \be
 \item The smoothness, or lack thereof,
in the way your car rides depends on your
shock absorbers.
Define the input to your shock absorbers and the output.
 \item What experiments would you perform to evaluate whether the shock
absorbers define a linear system?

 \item Could your shock absorbers be perfectly
linear over all input ranges?  Could they be linear
over any range?

 \item Draw an impulse response function for your
shock absorbers that defines the performance you
think is ideal in a car.  Why did you choose your design?
What are the tradeoffs?

 \ee

\item 
 \be

 \item  In the text we derived the response of a shift-invariant
linear system to a sinusoidal input.
Use the same general methods as in the text
to derive the response to a cosinusoidal input.
You will need to use the fact that $\cos(i + j) = \cos(i) \cos(j) + \sin(i) \sin(j)$.

 \item  In part (a) of this question,
you found that the response to $\cos(i)$ has the
general form $u \cos(i) + v \sin(j)$ for two real
numbers, $u$ and $v$.
We saw in the text that the response to $\sin(i)$ is
$a \cos(i) + b \sin(i)$ for two real numbers $a$ and $b$.
Compare the amplitude of the response to the sin
($a^2 + b^2$) with the amplitude of
the response to the cos ($u^2 + v^2$).

 \item Compare the phase shift of the
two responses, that is compare
$\tan ^{-1} ( \frac{b}{a} ) $ and $\tan ^ {-1} ( \frac{v}{u})$.

 \end{enumerate}

\item A $N$ by $N$ circulant matrix for
a shift-invariant linear system only contains
$N$ independent numbers.
These values can be found from the central column.
When we use a line as an input to a shift-invariant
linear system, we measure all $2N$ values in the response to 
a single stimulus.

 \be

 \item Suppose we use just one stimulus, a line,
as an input stimulus for a shift-invariant linear system.
How many independent output values will we measure?
\comment{N}

\item How many independent measurements do we obtain when we
use a sinusoidal input?
\comment{two}
\item How many different sinusoidal frequencies will we have
to measure before we obtain the same
number of measurements as from a line?
\comment{N/2}
\item How many measured values do we obtain when we use
an eigenfunction as an input stimulus?
\comment{one}
\item How many eigenfunctions must we measure before
we have $N$ independent values?
\comment{N}

\item Describe how you could use exponential patterns of light,
the eigenfunctions of shift-invariant linear systems,
to measure the retinal image.
Why is this idea not practical?

 \ee

\item Suppose the response of
a linear, shift-invariant system to
a cosinusoid is $a \sin(i) +  b \cos(i) $.
Using the fact that a sinusoid is a shifted copy of
a cosinusoid, express the response to a sinusoid in terms of $a$ and $b$.
You may need the fact that $\cos( - \pi /2 ) = 0$ 
and $\sin ( - \pi / 2 )  = -1$.
\comment{
sin(f) = cos(f - { \pi / 2} ) 
	-> A sin(f -  {\pi / 2}) + B cos(f -  {\pi / 2})  
	= A{ cos(f)sin(- {\pi / 2}) + cos(- {\pi / 2})sin(f)}
	  + B{ cos(f)cos(- {\pi / 2}) - sin(f) sin(- {\pi / 2})}
	= A{ cos(f) (-1) + 0 sin(f) }
	  + B{ cos(f) (0) - sin(f) (-1) }
 	= -A cos(f) + B sin(f)
	=  B sin(f) - A cos(f)
}

