Geisler has built a series of computational models
of the action of different components within the visual pathways.
By applying these models,
beginning with the stimulus
and moving through the
visual pathways,
to create a series of neural representations.
His calculations include computational
models of many elements of the visual pathways we have reviewed,
including the optics, the photoreceptor sampling mosaic,
the wavelength encoding, the ganglion cell mosaic, and so forth.
\begin{figure}
\centerline{
  \psfig{figure=../11tools/fig/idealObserver.ps,clip= ,height=3.5in}
}
\caption[Ideal Observer Analysis of Behavior]{
Ideal observer theory can be used
to analyze the loss of information
through the series of visual transformations.
}
\label{f11:idealObserver}
\end{figure}

We can perform the calculation
beginning with two different categories of stimuli.
At each stage of the calculation,
the different categories will be transformed into
a distribution of neurophysiological responses.
Using the computational model, we can 
estimate the likelihood of a particular
neurophysiological state being due to
to each of the two different stimuli.
From this information,
we can calculate the performance of an ideal observer
designed to discriminate between the two stimuli.
The ideal observer will make some errors;
we can both count their number and try
to analyze their properties.
By analyzing the performance of the ideal observer,
we may gain some insight into the information that
has is present, and the information that
has been lost, as we travel up the visual pathways
(Banks et al., 1987;  Geisler, 1984; 
Geisler and Davila, 1985; Geisler, 1987,1989).

The ideal observer performance does not often
come very close to the behavior of human observers.
For example, Banks et al. (1987)
report that an ideal observer
operating on all of the information available
to the photoreceptors would be about forty times
more sensitive than human observers at detecting spatial
frequency gratings above five cycles per degree.
They do point out the more encouraging observation
that the shape of the contrast sensitivity
function in the high frequency region
is approximately the same for the ideal and the human observer.
The differences in predicted level of performance
could be due to errors in the physiological model of
the optics, sampling, and so forth.
Or, they could be due to later factors that have not
yet been incorporated in the model.

The diffreence between the human and ideal observers, however,
is not the point.
The most important aspect of creating
and analyzing ideal observer performance
is that it forces us to think carefully
about (a) the information available to solve the task,
and (b) how that information is represented in the visual pathways.
These two questions are always present when we analyze performance
and behavior;
the ideal observer analysis provides us with a way to
organize our answers.



%%%%%%%%%%%%%%

Over the last few years the Bayesian methods have been
applied to the analysis of a much broader array
of perceptual tasks, beyond detection threshold
(see e.g. 
Ahumada and Watson, 1985;
Barlow, 1980;
Barlow Symmetric Papers;
Burgess, 1981,1988;
Geisler, 1984,1985,1987;
Kersten, 1987,1990;
Peli,,1985, 1990;
van Meeteren and Boogaard, 1973).
In these analyses, the performance of human observers
is compared with the performance of
an hypothetical observer who followed Bayes' Rule precisely.
An observer who follows Bayes' Rule
is called an {\em the ideal observer}, or a
{\em Bayes classifier}.

