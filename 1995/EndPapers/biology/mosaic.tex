\subsection*{Mosaic}

\be

\item Total number of cones in each retina, $5 \times 10^6$.
% \item Number of cones in human eye: 6.3-6.8 million
%: $4.08 \times 10^6 - 5.29 \times 10^6$ -- Curcio
% \item Number of cones per eye  6million
% "Human Photoreceptor Topography" 
% Curcio, Sloan, Kalina, and Hendrickson,
% Journal of Comparative Neurology 292:497-523 (1990), also see the
% paper by Osterburg, 1935
 


% Wandell

\item Foveal cone density:  $1.5 \times 10^4$.
% Wandell
% $7 \times 10^3 - 34 \times 10^3$, Curcio data
% $9.8 \times 10^3 - 32.4 \times 10^3 cones/mm^2$
%\item Maximum density of cones at foveal center: About 130,000,000/$mm^2$
%measured for seven individuals providing eight different retinas

\item Total number of rods in each retina: $10^8$.
% $78 \times 10^6 - 107 \times 10^6$
%\item Number of rods per (human) eye 120million
%\item Number of rods in human eye: 110-125 million,    Osterberg, 1935
% Wooding

\item Except in the fovea, where there are no rods,
ratio of rods to cones is 20 to 1.
\comment{
Human rods outnumber cones 20:1 in most
regions of the retina, except the fovea (Kolb,  VIs. Neuro 1991)
}
% Cited in Williams,Brainard from Yuodelis and Hendrickson

\item $\Blue$ cone separation: 10 arc min, one-dimensional sampling
density 6 $\Blue$ cones per deg, Nyquist sampling limit 3 cpd.
%\item $\Blue$ cones per degree of visual angle:  6
% Wandell

\item Combined $\Red$ and $\Green$ mosaic, foveal separation 0.5 arc
min, one-dimensional sampling density, 120 cones per deg, Nyquist
sampling limit 60 cpd.
%\item 30 arc sec = minimum separable; closest spacing of cones (approx.)
% O'Shea, Wandell, Schein

% \item Human retinal magnification:  290 $\mu$/deg
% Schein

\item[Macaque] retinal magnification:  210 $\mu$/deg
% Schein

\item 10 percent of 500nm light incident on photoreceptors is absorbed.
% Wooding

\item Peak density of cones: central 1-2 degrees, rods, 10-15 degrees off axis
% Wooding, everyone
% Osterburg

\item Peak absorption of Rhodopsin (rod pigment): 500nm
% Wooding

\item Ratio of $\Red$ and $\Green$ cones to $\Blue$ cones: 100
% Wooding

\ee


