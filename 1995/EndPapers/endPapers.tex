\documentstyle[12pt]{article}
% Documentstyle included in preamble.tex
%
\input{/u/brian/lib/tex/standardBook}
\begin{document}

\def\sfrac#1#2{{\textstyle \frac{#1}{#2}}}   
\def\mat#1{{\bf #1}}

%
%	This resides in /usr/share/src/Tex/lib/inputs
%
\input{psfig}

% \newcommand{\psfigpath}{/home/brian/images/general}
%
% 	I think this sets it up for draft mode rather than actually
% 	including the stuff
%\psdraft


%
%	All the math includes
%
%	From Chapter on Image Formation
%
%

%%%%%%%%%%%%%%%%%%%%%%%%%%%%%%%%
%
%
%	General vector and matrix notation
%
\newcommand{\vect}[1]{\mbox{${\bf #1}$}}
\newcommand{\vecti}[2]{\mbox{${#1}_{#2}$}}
\newcommand{\vectij}[3]{\mbox{${#1}_{#2,#3}$}}
\newcommand{\matr}[1]{\mbox{${\bf #1}$}}

\newcommand{\lin}{\mbox{{ \boldmath $L$ } }}
\newcommand{\twopN}{\mbox{${\frac{2 pi}{N}}$}}
\newcommand{\sumoneNn} {\sum_{n = 1}^{N}}
\newcommand{\sumoneNi} {\sum_{i = 1}^{N}}
\newcommand{\sumoneNj} {\sum_{j = 1}^{N}}
\newcommand{\sumoneNk} {\sum_{k = 1}^{N}}
\newcommand{\sumoneNf} {\sum_{f = 1}^{N}}
\newcommand{\sumoneNfx}{\sum_{fx = 1}^{N}}
\newcommand{\sumoneNfy}{\sum_{fy = 1}^{N}}
\newcommand{\sumzeroNn}{\sum_{n = 0}^{N - 1}}
\newcommand{\sumzeroNi}{\sum_{i = 0}^{N - 1}}
\newcommand{\sumzeroNj}{\sum_{j = 0}^{N - 1}}
\newcommand{\sumzeroNk}{{\sum_{k = 0}^{N - 1}}}
\newcommand{\sumzeroNf}{\sum_{f = 0}^{N - 1}}
\newcommand{\sumzeroNfx}{\sum_{fx = 0}^{N - 1}}
\newcommand{\sumzeroNfy}{\sum_{fy = 0}^{N - 1}}
\newcommand{\sumoverNi} {\sum_{i = \langle N \rangle}}
\newcommand{\sumoverNj} {\sum_{j = \langle N \rangle}}
\newcommand{\sumoverNk} {\sum_{k = \langle N \rangle}}

\newcommand{\Dzero}{\mbox{$(1,0,0, \ldots ,0)^t$}}
\newcommand{\Done}{\mbox{$(0,1,0, \ldots ,0)^t$}}
\newcommand{\DN}{\mbox{$(0,0,...,0,1)^t$}}
\newcommand{\Di}{\mbox{$(0,..,1,...0)^t$}}

%	Starting with Chapter on Optics
%
\newcommand{\lspread}{\mbox {$\bf l$} }
\newcommand{\lspreadi}[1]{\mbox{$l_{#1}$}}
\newcommand{\lspreadhat}{\mbox {$\hat{\bf l}$}}
\newcommand{\lspreadhati}[1]{\mbox {$\hat{l}_{#1}$}}
\newcommand{\Resmat}{\mbox {$\bf R$} }
\newcommand{\resvec}{\mbox {$\bf r$} }
\newcommand{\resveci}[1]{\mbox{$r_{#1}$}}
\newcommand{\resvechat}{\mbox {\bf $\hat{r}$}}
\newcommand{\resvechati}[1]{\mbox {$\hat{r}_{#1}$}}
\newcommand{\pixmat}{\mbox {$\bf P$} }
\newcommand{\pixvec}{\mbox {$\bf p$} }
\newcommand{\pixveci}[1]{\mbox{$p_{#1}$}}
\newcommand{\pixvechat}{\mbox {$\bf \hat{p}$}}
\newcommand{\pixvechati}[1]{\mbox{$\hat{p}_{#1}$}}
\newcommand{\Optics}{\mbox{$\bf O$}}
\newcommand{\pfiN}{\mbox{$\frac {2 \pi f i}{ N }$}}
\newcommand{\SfNi}{\mbox{$\sin  ( \frac {2 \pi f i}{ N } )$}}
\newcommand{\SfnNi}{\mbox{$\sin  ( - \frac {2 \pi f i}{ N } )$}}
\newcommand{\SfiNk}{\mbox{$\sin ( \frac {2 \pi f_i k}{ N } )$}}
\newcommand{\SfjNk}{\mbox{$\sin ( \frac{2 \pi f_j k}{ N } )$}}
\newcommand{\CfNi}{\mbox{$\cos (  \frac{2 \pi f i}{ N } )$}}
\newcommand{\CfiNk}{\mbox{$\cos ( \frac{2 \pi f_i k}{ N })$}}
\newcommand{\CfjNk}{\mbox{$\cos ( \frac{2 \pi f_j k}{ N } )$}}
\newcommand{\SfNj}{\mbox{$\sin (  \frac{2 \pi f j}{ N } )$}}
\newcommand{\CfNj}{\mbox{$\cos (  \frac{2 \pi f j}{ N })$}}
\newcommand{\SfNk}{\mbox{$\sin (  \frac{2 \pi f k}{ N })$}}
\newcommand{\SfnNk}{\mbox{$\sin ( - \frac{2 \pi f k}{ N })$}}
\newcommand{\CfNk}{\mbox{$\cos (  \frac{2 \pi f k}{ N })$}}
\newcommand{\SfNij}{\mbox{$\sin ( \frac{2 \pi f (i + j)}{ N } )$}}
\newcommand{\SfoneNk}{\mbox{$\sin ( \frac{2 \pi f_1 k}{ N } )$}}
\newcommand{\SftwoNk}{\mbox{$\sin ( \frac{2 \pi f_2 k}{ N } )$}}
\newcommand{\CftwoNk}{\mbox{$\cos ( \frac{2 \pi f_2 k}{ N } )$}}
\newcommand{\matCos}{\mbox{$ \bf C$}}
\newcommand{\matSin}{\mbox{$ \bf S$}}

%	From chapter on color
%

\newcommand{\Red}{\mbox{L}}
\newcommand{\Green}{\mbox{M}}
\newcommand{\Blue}{\mbox{S}}
\newcommand{\redw}{\Red ( \lambda )}
\newcommand{\greenw}{\Green ( \lambda )}
\newcommand{\bluew}{\Blue ( \lambda )}
\newcommand{\evec}{\mbox{$\bf e$}}
\newcommand{\eveci}[1]{\mbox{$e_{#1}$}}
\newcommand{\er}{e_r}
\newcommand{\eg}{e_g}
\newcommand{\eb}{e_b}
\newcommand{\test}{\mbox{${\bf t}$}}
\newcommand{\testi}[1]{\mbox{$t_{#1}$}}
\newcommand{\testw}{t ( \lambda )}
\newcommand{\lms}{\mbox{$\bf r$}}
\newcommand{\Iodopsin}{\mbox{$\bf B$}}
\newcommand{\Rhodopsin}{\mbox{$\bf A$}}
\newcommand{\Rhodopsinw}{\mbox{$\bf A(\lambda)$}}
\newcommand{\Scotopic}{\mbox{$\bf R$}}
\newcommand{\Scotopicw}{\mbox{$R ( \lambda )$}}
\newcommand{\Scotopici}[1]{R_{#1}}
\newcommand{\primaryint}{\mbox{$\bf e$}}
\newcommand{\primaryinti}[1]{\mbox{$e_{#1}$}}
\newcommand{\primary}{\mbox{$\bf p$}}
\newcommand{\primaryi}[1]{\mbox{${\bf p}_{#1}$}}
\newcommand{\Primarymat}{\mbox{$\bf P$}}
\newcommand{\Photopic}{\mbox{\bf C}}
\newcommand{\Monitor}{\mbox{$ \bf M $}}
\newcommand{\Calibration}{\mbox{$\matr{H}$}}
\newcommand{\monitori}[1]{\mbox{$\bf m_{#1}$}}
\newcommand{\Sensor}{\mbox{\bf S}}
\newcommand{\current}{\mbox{\bf i}}
\newcommand{\currentt}{\mbox{$i ( t )$} }
\newcommand{\nl}{\mbox{$n_\lambda$}}

%
%	From chapter  on physiology
%
\newcommand{\Pbeta}{P_\beta}
\newcommand{ \amp} { Amplitude }
\newcommand{ \sumxx} { \sum_{x = 0} to {x = N} }
\newcommand{ \sumtt} { \sum_{t = 0} to {t = N} }
\newcommand{ \Aftx} { A ( ft , x )}
\newcommand{ \BftxC} { {  W ( ft , C(x) ) } }
\newcommand{ \dt} { { delta t } }
\newcommand{ \fx} {f_{x}}
\newcommand{ \ft} {f_{t}}
\newcommand{ \phix} {p_{x} }
\newcommand{ \phit} {p_{t} }
\newcommand{ \contx} { a_{x} }
\newcommand{ \contt} { a_{t} }
\newcommand{ \intensityi}[1]{\mbox{$i_{#1}$}}
\newcommand{\mean}{\mbox{$m$}}
\newcommand{\pspread}{\mbox {$\bf p$} }
\newcommand{\pspreadi}[1]{\mbox{$p_{#1}$}}
\newcommand{\pspreadhat}{\mbox {\bf $\hat{p}$}}
\newcommand{\pspreadhati}[1]{\mbox {$\hat{p}_{#1}$}}
\newcommand{ \RF} {\mbox{ \bf R}}	%Receptive field
\newcommand{ \linex} {\delta(x)}
\newcommand{ \responsemat} {\bf F}
\newcommand{ \response} {\bf f}
\newcommand{ \responsei} {\mbox{$f_{i}$} }
\newcommand{ \responset} {f ( t ) }
\newcommand{ \contrast}{\mbox{$a$}}
\newcommand{ \contrasti}[1]{\mbox{$a_{#1}$}}
\newcommand{ \contrastvec}{\mbox{$\bf a$}}
\newcommand{ \contrastmat}{\mbox{$\bf A$}}
\newcommand{ \onecontrastmat} {\bf U}
\newcommand{ \LW} {\bf L}
\newcommand{ \As} {A_{s}}
\newcommand{ \Bs} {B_{s}}
\newcommand{ \RFs} {RF_{s}}
\newcommand{ \Ac} {A_{c}}
\newcommand{ \deltai} {(0,..,1,...0)^{t}}

%
%	From chapter on Pattern Vision
%
\newcommand{\svec}{\mbox{$\bf s$}}
\newcommand{\wvec} { \mbox{$\bf w$}}
\newcommand{\stim} { \mbox{$\bf s$} }
\newcommand{\stimi}[1]{ \mbox{$\bf s{_#1}$} }
\newcommand{\neural}{\mbox{$\bf n$}}		%Neural image vector
\newcommand{\neuralx}[1]{\mbox{$n_{#1}$}}
\newcommand{\length}{\mbox{$d$}}
\newcommand{\lengthi}[1]{\mbox{$d_{#1}$}}
\newcommand{\sinlengthi}[1]{\mbox{$a_{#1}$}}

%
%	From chapter on multiresolution

\newcommand{\gest}[1]{\mbox{$\hat{g}_{#1}$}}

%	From the chapter on Color Appearance
%
\newcommand{\recSens}[1]{\mbox{$R_{#1}$}}
\newcommand{\recResp}{\mbox{$\bf r$}}
\newcommand{\recRespi}[1]{\mbox{$r_{#1}$}}
\newcommand{\sensorMat}{\mbox{$\bf S$}}

\newcommand{\colsig}[1]{\mbox{$c(#1)$}}
\newcommand{\colsigi}[1]{\mbox{$c_{#1}$}}

\newcommand{\ill}[1]{\mbox{$e(#1)$}}
\newcommand{\illhat}[1]{\mbox{$\hat{e}(#1)$}}
\newcommand{\illi}[2]{\mbox{$e_{#1}(#2)$}}
\newcommand{\illvec}{\mbox{$\bf e$}}

\newcommand{\illveci}[1]{\mbox{$\bf e_{#1}$}}
\newcommand{\illBasis}{\mbox{$\bf B_e$}}
\newcommand{\illBasisi}[2]{\mbox{$E_{#1}(#2)$}}
\newcommand{\illCoef}{\mbox{${\bf \omega}$}}
\newcommand{\illCoefi}[1]{\mbox{$\omega_{#1}$}}
\newcommand{\illMat}[1]{\mbox{$\Lambda_{#1}$}}
\newcommand{\illMatinv}[1]{\mbox{$\Lambda^{-1}_{#1}$}}

\newcommand{\surf}[1]{\mbox{$s(#1)$}}
\newcommand{\surfvec}{\mbox{$\bf s$}}
\newcommand{\surfveci}[1]{\mbox{$s_{#1}$}}
\newcommand{\surCoef}{\mbox{${\bf \sigma}$}}
\newcommand{\surCoefi}[1]{\mbox{$\sigma_{#1}$}}
\newcommand{\surBasis}{\mbox{$\bf B_s$}}
\newcommand{\surBasisi}[2]{\mbox{$S_{#1}(#2)$}}


%	Motion chapter
%
\newcommand{\parder}[2]{\mbox{$\frac{\partial #1}{\partial #2}$}}

%	Appendix A:  Shift-invariance
%
\newcommand{\Cyclic}{\mbox{$\bf C$}}
\newcommand{\Cyclichat}{\mbox{ $\hat{{\bf C}}$}}
\newcommand{\Cyclichati}[1]{\mbox{ $\hat{{C_{#1}}}$}}

%	Appendix B:  Color calibration
%

%	Appendix C:  Classification
%
\newcommand{\data}{ \mbox{${\bf d}$}}
\newcommand{\stimA}{\mbox{$A$}}
\newcommand{\stimB}{\mbox{$B$}}
\newcommand{\muAi}{\mbox{$\mu_{A,i}$}}
\newcommand{\muBi}{\mbox{$\mu_{B,i}$}}


%
%	From chapter on Tools
%
\newcommand{\dvec}{ \bf d }
\newcommand{\sd}{\sigma}
\newcommand{\muA}{\mu_{A} }
\newcommand{\sdA}{\sigma_{A} }
\newcommand{\muB}{\mu_{B}}
\newcommand{\sdB}{\sigma_{B}}
\newcommand{\mui}{\mu_{i} }
\newcommand{\sdi}{\sigma_{i} }

%	Appendix D:  Signal Estimation
%

%	Appendix E:  Motion flow
%
\newcommand{\motFlow}[2]{\mbox{${\bf m}(#1,#2)$}}


\newcommand{\cdmsquared}{\mbox{$ cd / m^2$} }
\newcommand{\mm}{\mbox{$mm$} }
\newcommand{\squaremm}{\mbox{$mm^2$} }
\newcommand{\cubicmm}{\mbox{$mm^3$} }
\newcommand{\squaredeg}{\mbox{$deg^2$} }

\begin{document}
\singlespace
\vspace{1.6in}

\subsection*{Units}

\be
\item Radiometric units (e.g., radiance) represent physical energy
(e.g., photons/${m^2}$/nm) 

\item Colorimetric units (e.g. luminance) measure radiance adjusted
visual wavelength sensitivity (e.g., \cdmsquared); scotopic luminance
units are proportional to the number of rod absorptions;  photopic
luminance units are proportional to a weighted sum of the $\Red$ and
$\Green$ cone absorptions
% Wandell

\item Typical ambient luminance levels (in \cdmsquared):
starlight $ 10^{-3}$; moonlight $  10^{-1}$; indoor
lighting $ 10^2$; sunlight $ 10^5$; max intensity of
common CRT monitors, $10^2$
%Source: Hood \& Finkelstein, Ch. 5 in some book from which I have an
%unlabeled reprint.  Maybe it's: Handbook of perception and human performance /
%editors, Kenneth R. Boff, Lloyd Kaufman, James P. Thomas. New York :
%Wiley, c1986.]
% Ben Backus

\item One Troland (Td) is produced on the retina when the eye is looking at a
surface of 1 $\cdmsquared$ through a pupil of area 1 \squaremm.
%Steve Burns

%\item Over a wide intensity range the pupil diameter is near 3.6 mm,
%the pupil area is $10 mm^2$, and Tds = $10 \times $ \cdmsquared 
% Chris Tyler

\item Lens focal length $f$ (meters); lens power = $1/f$ (diopters)

\item  Conversion of linear units (X) to decibels: Y = $20 \log_{10} (X)$
%a change of 6 dB is a factor of 2
%20 dB is 1 log unit
% Tim Meese

\item A change of 0.3 $\log_{10}$ units is a factor of 2, or 6 dB
%Tyler

\ee

\subsection*{Image Formation}

\be

\item The eyes are 6 cm apart and half-way down the head
%Ben Backus, Chris Tyler

\item Visual angle of the sun or moon = 0.5 deg

\item At arm's length: thumbnail = 1.5 deg; thumb joint= 2.0 deg; fist
= 8-10 deg
% Davida Teller;
% O'Shea, R. P. (1991). Thumb's rule tested: Visual angle of thumb's
% width is about 2 deg. Perception, 20, 415-418.
% Wandell

\item Monocular visual field measured from central fixation: 
160 deg (w) $\times$ 175 deg (h)
%60 deg nasal; 100 deg temporal; 60 deg above; 75 deg below; 
%total area ($2 \times 10^4 \squaredeg$)
%David Wooding

\item Binocular visual field measured from central fixation: 200 deg
(w) $\times$ 135 deg (h) 

\item Region of binocular overlap: 120 deg (w) $\times$ 135 deg (h)
% David Wooding
%(see Grigsby and Tsou (1994), Vision  Research 34, 2841-2848)

\item Range of pupil diameters:  2mm -8mm.
% Wandell, S. Burns

\item Refractive indices: air $1.000$; glass $1.520$; water $1.333$;
cornea $1.376$
% Wandell

\item Optical power (diopters): cornea, 43; lens, 20 (relaxed); whole eye, 60

\item Change in power due to accommodation, 8 diopters

\item Axial chromatic aberration over the visible spectrum: 2 diopters
% (Howarth and Bradley, Vision Research 26, 361-366, 1986;
%Wald and Griffin 1947, JOSA, 37, 321-336).
% Peter Howarth
% Source:  Davson, H. (Ed.) The Eye, vol. 4, page 105 (1962), Academic
% Press. This is the Gullstrand schematic eye.
% Brian Brown, L. Spillmann, Wandell

\ee

\subsection*{Retina}

\be

\item Retinal size: 5 cm $\times$ 5 cm; 0.4 mm thick

\item One degree of visual angle = 0.3 mm on the retina
% Boynton, Tyler, Wandell

\item Number of cones in each retina: $5 \times 10^6$

\item Number of rods in each retina: $10^8$
%: $4.08 \times 10^6 - 5.29 \times 10^6$ -- Curcio
% "Human Photoreceptor Topography" 
% Curcio, Sloan, Kalina, and Hendrickson,
% Journal of Comparative Neurology 292:497-523 (1990), also see the
% paper by Osterburg, 1935
% $78 \times 10^6 - 107 \times 10^6$
%\item Number of rods per (human) eye 120million
%\item Number of rods in human eye: 110-125 million,    Osterberg, 1935
% Wandell, Wooding

\item Diameter of the fovea: 1.5 mm (5.2 deg); rod-free fovea: 0.5 mm
(1.7 deg); foveola (rod-free, capillary-free fovea): 0.3 mm (1 deg);
size of the optic nerve head: 1.5 mm $\times$ by 2.1 mm (5 deg (w)
$\times$ 7 deg (h))
 
\item Peak cone density:  $1.6 \times 10^5$ cones/\squaremm;
%\item Thickness of retina: 350 microns
% Wooding, Wandell, Boynton
% Wandell
% From Curcio, Science, 1987, fig 3. 10^5.
% Also, notice that if there are 120 cones/deg in the foveola,
% then 120*120 = 14,400;  1 deg = .3 mm by .3 mm = .09 mm
% 14,400 cones / .09 \squaremm  = 160,000 cones/squaremm;
% This is almost identical to Curcio's numbers.

\item Foveal cone size: 1-4 $\mu$ (diameter) $\times$ 50-80 $\mu$ (length);
extrafoveal cone size: 4-10 $\mu$ (diameter) $\times$ 40 $\mu$ (length)
% Wandell, Wooding, Tyler

\item Size of rods near fovea: 1 $\mu$ (diameter) $\times$ 60 $\mu$ (length)
% Wooding Tyler

\item $\Blue$ cone spacing (foveal): 10 arc min
% one-dimensional sampling density: 6 $\Blue$ cones per deg
%\item $\Blue$ cones per degree of visual angle:  6
% Wandell

\item $\Red$ and $\Green$ spacing (foveal): 0.5 arc min
% one-dimensional sampling density: 120 cones per deg;
% O'Shea, Wandell, Schein

\item Number of ($\Red$ + $\Green$) / Number of $\Blue$ cones: 100
% Wooding

\item 1.5 $\times 10^6$ optic nerve fibers/retina; ratio of receptors to
ganglion cell in fovea 1:3; ratio of receptors to ganglion cells for
whole retina, 125:1
\comment{  (Schein, 1988;  Wassle et al. 1989a, cited in Kolb, 1991)}
\comment{(Sherman and Koch, p. 249)}
% Schein, Ross

\ee

\subsection*{Cortex}

\be

\item Area of entire cortex: $1.3 \times 10^5$ \squaremm; 1.7 mm thick 
% (1.5 square feet)
%, weight 0.25 kg
%Wandell, Kersten
%(Hubel in Rock Readings);  Cherniak says 1.6 \times 10^6

\item Total number of cortical neurons: $10^{10}$;
density: $10^5$ neurons / \cubicmm
%Kersten, Wandell, Shadlen, Tyler, Britten
%(Bailey and von Bonin 1951)
% Cited in (Valverde, Peters and Jones, p. 210)
%(O'Kusky and Colonnier, J Comp Neurol 1982, 210:278) 
%(see also, Peters' chapter in Cerebral cortex, Vol 6, Jones and Peters
%eds. 1987, New York, Plenum) 
% (Hubel and W/Rock)
% (O'Kusky and Colonnier, J Comp Neurol 1982, 210:278)
% (see also, Peters' chapter in Cerebral cortex, Vol 6, Joes and Peters
% eds. 1987, New York, Plenum)

\item Synapses: $ 5 \times 10^ 8$ synapses / \cubicmm; 
$4 \times 10^3$ synapse/neuron; 
% Kersten, Shadlen
% see  Cherniak,  J. of Cog. Neurosc., 1990, vol 2., pp 58-68

\item Axons: 3 kilometers /\cubicmm
%Shadlen
%These and several other fun numbers can be found in
%Stevens CF, 'What form should a cortical theory take' in Large-Scale
%Neuronal Theories of the Brain, Koch C and Davis JL eds., MIT press,
%Cambridge, Mass., 1994, pp.239-255.

%\item 10 x more fibers from cortex to thalamus than thalamocortical
%(Sherman and Koch, p. 253).  (Hubel)
%\item Number of fibers from thalamus to cortex: $4 \times 10^{5}$
%(Sherman and Koch, p. 253).  (Hubel)

\item Number of corpus callosum fibers:  $0.5 \times 10^{9}$
%Wandell

\item Number of macaque visual areas:  30
%; interconnections between macaque visual areas: 300 
%(Felleman and van Essen, 1991)

\item Size of each area V1: 3cm by 8 cm
%together they occupy 1.5\% of human cortex
% Tyler, Wandell
%Crick and Koch in Sci Am. on Brain 1992  
%44mm x 44mm.
%look up in Horton and Hoyt;
%Steve has a review article on this.

\item Half of area V1 represents the central 10 deg (2\% of the
visual field)
%Wandell

\item Width of a human ocular dominance column 0.5-1.0 $mm$; width of
a macaque ocular dominance column 0.3 $mm$.
%Wandell (Horton and Hoyt)

\ee

\subsection*{Sensitivity}

\be

\item Minimum number of absorptions for: scotopic detection 1-5;
detectable electrical excitation of a rod 1; photopic detection 10-15
% Wandell

\item Following exposure to a sunny day, dark adaptation to a
moonless night involves: 10 minutes (photopic); 40 minutes (scotopic);
change in visual sensitivity 6 $\log_{10}$ units
%Tyler, Wooding
%Ross

\item Highest detectable spatial frequency at high ambient light levels,
50-60 cpd; low ambient light levels, 20-30 cpd
%Wandell, Teller

\item The contrast threshold ($\Delta L / L$) for a static edge at photopic
luminances is 1\%.  
%Pelli
%This predicts the visibility of a wide variety of
%large sharp edged objects, including letters.  Robson, J. G. (1993)
%Contrast sensitivity: One hundred years of clinical measurement. In
%R. Shapley and D. Man-Kit Lam (Eds.), Contrast Sensitivity
%(pp. 253-266), Cambridge, MA: MIT Press.

\item Highest detectable temporal frequency:  high ambient large
field, 80 Hz; low ambient, large field 40 Hz.
% Wandell, Tyler

\item Typical localization threshold: 6 arc sec ($0.5 \mu$)
%  one-fifth the center-to-center spacing of foveal cones
% 0.3/120 = center-to-center spacing of foveal cones
% 6 arc sec = 0.3 / 600
% Ratio is 1/5

\item Minimum temporal separation needed to discriminate two small, brief
light pulses from a single equal-energy pulse: 15-20 ms
%Nilson
%(photopic) 
% (Vision Res '79, ARVO '92).

\item Stereoscopic depth discrimination: step threshold, 3 arc sec;
point threshold, 30 arc sec
% Backus
% [Stevenson, S. B., Cormack, L. K., \& Schor, C. M. (1989).
% Hyperacuity, superresolution and gap resolution in human stereopsis.
% Vision Res, 29(11), 1597-605.]

\ee

\subsection*{Color}

\be

\item Visible spectrum: 370-730 nm

\item Peak wavelength sensitivity: 507nm (scotopic) and 555 nm (photopic)
% photopic and scotopic sensitivity curves cross at 528nm
%Backus, R. Boynton

\item Spectral equilibrium hues: 475 nm (blue),
500 nm (green), 575 nm (yellow), no spectral equilibrium red
%Varner

\item Number of basic English color names: 11
% G. Boynton
% (Boynton, Robert M. ; Olson, Conrad X. Vision Research.  1990 Vol 30(9)
% 1311-1317.) 

\item Incidence of:
anomalous trichromacy, $10^{-2}$ (male), $10^{-4}$ (female);
protanopia and deuteranopia, $10^{-2}$ (male), $10^{-4}$ (female);
tritanopia, $10^{-4}$; rod monochromac, $10^{-4}$; cone monochromacy,
$10^{-5}$
%Wooding

\ee

\end{document}

